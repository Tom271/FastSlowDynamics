\subsection{Oscillations}\label{sec: MMO Oscilaltions} %Need this to reference the oscilatory behavious of the system
In this section we consider Mixed Mode Oscillations (MMOs) in fast-slow systems.

\begin{definition}{Mixed Mode Oscillations}
+++ working on it... not good yet+++
A mixed mode oscillation is an orbit $\gamma$, which traces out small amplitude oscillations (SAOs) as well as large scale oscillations (LAOs).
The large and small amplitude oscillations are clearly separated in the time series and their reoccurence can be periodic.
The signiture of an MMO is expressed as $L_1^{s_1}L_2^{s_2}...$ expressing that $L_1$ number of LAOs are followed by $s_1$ SAOs.
\end{definition}

The cases of MMOs considered here are MMOs associated with folded nodes as well as folded saddle-nodes of type 2, that are associated to singular hopf bifurcations.
+++++++++++needs better intro.++++++++++++

\subsection{ Folded Nodes}

\subsection{Singular Hopf Bifurcation}


\begin{definition}{Hopf Bifurcation}

\end{definition}



The normal form of Section (++ref toms section+++) is transformed by adding higher-order terms to the $z$ equation. It then becomes
\begin{align*}
\epsilon \dot{x} &= y - x^2, \\
\dot{y} &= z - x \\
\dot{z} &= - \nu -ax - by - cz.
\end{align*}

We then consider a coordinate transformation and time rescaling of the form 
\begin{align*}
x = \epsilon^{1/2}\overline{x}, \ \ \ y= \epsilon \overline{y},  \ \ \ z = \epsilon^{1/2} \overline{z},\ \ \  t= \epsilon^{1/2} \overline{t}.
\end{align*}
Then the system becomes
\begin{align*}
\overline{x}' &= \overline{y} - \overline{x}^2, \\
\overline{y}' &= \overline{z} - \overline{x}, \\
\overline{z}' &= - \nu - \epsilon^{1/2} a \overline{x} - \epsilon b \overline{y} - \epsilon^{1/2} c \overline{z}.
\end{align*}




much later on: we add a cubic term to get an s shape. global return:
\begin{align*}
\epsilon \dot{x} &= y - x^2 - x^3, \\
\dot{y} &= z - x, \\
\dot{z} &= -\nu -ax -by -cz.
\end{align*}








