\subsection{Oscillations}\label{sec: MMO Oscilaltions} %Need this to reference the oscilatory behavious of the system
In this section we consider Mixed Mode Oscillations (MMOs) in fast-slow systems.
++++ add that we consider the work from the \citet{MMO} paper unless indicated otherwise

\begin{definition}{Mixed Mode Oscillations}
	+++ working on it... not good yet+++ \citet{MMO} \citet{kuehn}
	A mixed mode oscillation is an orbit $\gamma$, which traces out small amplitude oscillations (SAOs) as well as large scale oscillations (LAOs).
	The large and small amplitude oscillations are clearly separated in the time series and their reoccurence can be periodic.
	The signiture of an MMO is expressed as $L_1^{s_1}L_2^{s_2}...$ expressing that $L_1$ number of LAOs are followed by $s_1$ SAOs.
\end{definition}

The cases of MMOs considered here are MMOs associated with folded nodes as well as folded saddle-nodes of type 2, that are associated to singular hopf bifurcations.
+++++++++++needs better intro.++++++++++++

\subsection{ Folded Nodes}
The singularity considered in this section is a folded node, which is an equilibrium of the reduced system. Note that it is only defined on $S$, the critical manifold and only for the slow flow. There is no global equilibrium, wich will become apparent in this section.
The normal form considered for analysing the folded node singularity is in terms of the space variables $(u,v,w)$, and given by:

\begin{align*}
\epsilon \dot{u} &= v - u^2\\
\dot{v} &= w-u\\
\dot{w} &= - \nu
\end{align*}

Then the system can be transformed using the following coordinate and time transformation:
\begin{align}\label{normalform1}
u= \frac{x}{(1+ \mu)^{1/2}}, \ \ \ v &= \frac{y}{(1 + \mu)}, \ \ \  w= -\frac{z}{ (1+\mu)^{3/2}}  \\
\tau &= \frac{\tau_1}{\sqrt{1 + \mu}},
\end{align}
where $\tau_1$ is the original time variable and $\tau$ is the transformed time variable.
Then $\frac{d \tau}{d \tau_1} = \frac{1}{\sqrt{1+ \mu}}$, and the system becomes:
\begin{align}\label{normalform2}
\epsilon \dot{x} &= y - x^2\\
\dot{y} &=- z -(\mu +1)x \\
\dot{z} &= \nu (1 + \mu)^2
\end{align}
where $\mu$ is the eigenvalue ratio from before.
This is nearly in the form presented in \citet{MMO}, however, the $z$ equation there is witten purely in terms of $\mu$ as $\dot{z} = \frac{1}{2} \mu$.
Equating these two representations yields a relationship between $\mu$ and $\nu$:
\begin{align*}
&\nu (1 + \mu)^2 = \frac{1}{2} \mu \\
&\Rightarrow \nu = \frac{\mu}{ 2 (1+ \mu)^2},
\end{align*}
and this is equivalent to 
\begin{align*}
\mu = \frac{ -1 + \sqrt{1 - 8\nu}}{-1 - \sqrt{ 1- 8 \nu}},
\end{align*}
since $0< \mu < 1$ and $\mu \in \mathbf{R}$. (++++ unsure about reasoning+++)

Note here that the reason that no global equilibrium exists is because \ref{normalform1} can only have an equilibrium if $\dot{w} =0$. This would imply that $\nu=0$, however, as the previous calculations have shown, $\nu$ is dependent on the eigenvalue ratio $\mu$. Since $\mu \neq 0$ for the folded node, as will be demonstrated below, $\nu$ cannot be zero.
It is now of interest to verify the location of the folded singularity at the origin, and therefore derive the reduced system as well as the eigenvalues for the reduced problem.
Consider equation (\ref{normalform2}) and define $\dot{x}:=f$ as before. The reduced problem, as $\epsilon \to 0$ becomes $f= y-x^2 =0$, and therefore the critical manifold is defined as $S:= \{ (x,y,z) : y=x^2\}$, which is an S shaped two dimensional plane.
Now that $f$ is defined explicitly, we can check the nondeceneracy conditions for a folded singularity, as presented in (\ref{eq: non-degeneracy 3d system}) and get the follwing results:
\begin{align*}
&f(x,y,z,,\mu, \epsilon) = 0\\
&\Rightarrow y=x^2\\
\\
&\frac{\partial f}{\partial x} (x,y,z,\mu,\epsilon) = 2x = 0\\
&\Rightarrow x=0 \Rightarrow y=0 \\ 
&\frac{\partial^2 f}{\partial x^2}(x,y,z,\mu,\epsilon)  = 2 \neq 0\\
&D_{(y,z)}f= (1,0) \textrm{ full rank one}.
\end{align*}
This shows that there exists a fold line $L:=(0,0,z)$ on the slow manifold $S$.
In order to determine at which value of $z$ the folded node singularity is located, we have to consider the reduced system of (\ref{normalform2}), where we replace $\nu (1 + \mu)^2$ with $\frac{1}{2} \mu$ for convenience. The aim is to find an equilibrium of the reduced problem, since we know from the theory discussed that the folded singularity is an equilibrium of the slow flow.
The reduced problem is:
\begin{align}\label{normalform2red}
0 &= y - x^2 :=f\\
\dot{y} &=- z -(\mu +1)x \\
\dot{z} &=\frac{1}{2} \mu 
\end{align}
Therefore, the slow flow is derived, analogous to Section (I+++ i guess VDP but also after that+++). 
First, the equation $f=0$ is considered and it is noted that we can take the derivative with respect to the time variable to get 
\begin{align} \label{yxderivrel}
\dot{y} = 2x \dot{x},
\end{align}
 and this can be rearranged to give an expression for the dynamics in $x$ on the slow manifold:
\begin{align*}
\dot{x}= \frac{\dot{y}}{2x},
\end{align*}
which is singular for $x=0$, which coinsides with the fold line.
This expression can be desingularised by rescaling time in the whole reduced system by a factor of $2x$. This results in
\begin{align*}
\dot{x} = -(\mu+1)x - z\\
\dot{y} = - 2x (\mu +1) -2xz\\
\dot{z} = x \mu,
\end{align*}
however, it can be noted that the equation for $y$ can be ommited, since the change in $y$ is directly related to the change in $x$ by a factor of $2x$ as stated in equation (\ref{yxderivrel})(+++mention CMT?+++). Therefore, the reduced dynamics can be sufficiently described by
\begin{align*}
\dot{x} = -(\mu+1)x - z\\
\dot{z} = x \mu.
\end{align*}
Now, following the theory for folded singularities, the folded node has to satisfy the condition (+++add name of the condition and generalized statement of it+++)
\begin{align*}
& -(\mu+1)x - z=0 |_{(0,0,z)}\\
&\Rightarrow z=0,
\end{align*}
which leads to the conclusion that the folded singularity, defined on the slow manifold for $\epsilon \to 0$, is given by $(0,0,0)$, as expected.
The next step of the analysis is to verify that the folded singularity at the origin is indeed a folded node.
As discussed in Section \ref{sec: threedimfolds}, the classification of the singularities is determined by the eigenvalues of the reduced system. Therefore, the next step is calculating these eigenvalues.














\begin{theorem}[\textbf{Width of Rotational Sectors}][\citealp{MMO}]
Consider system (\ref{somegenericthreedim})and assume it has a folded-node singularity. At an $O(1)$ distance from the fold curve, all secondary canards are in an $O(\epsilon^{(1- \mu)/2)})$ neighbourhood of the primary strong canard. Hence, the width of the  rotational sectors $I_i, 1 \leq i \leq k$, is $O(\epsilon^{(1- \mu)/2)})$ and the width of sector $I_{k+1}$ is $O(1)$.
\end{theorem}



+++++++++++Return Mechanism++++++


\begin{theorem}[\textbf{Generic $1^{k+1}$ MMOs}][\citealp{MMO}] \label{MMOsigk1}
Consider system (\ref{somegenericthreedim}) with the following assumptions:
\begin{enumerate}
\item Assume that $ 0 < \epsilon \ll 1$ is sufficiently small, $\epsilon^{1/2} \ll \mu$, and $k \in \mathbf{N}$ is such that $2k + 1 < \mu^{-1} < 2k + 3$.
\item The critical manifold $S$ is (locally) a folded surface.
\item The corresponding reduced problem possesses a folded-node singularity.
\item There exists a candidate periodic orbit, which consists of fast fibres of the layer problem, a global return segment, and a segment on $S^a$ within the funnel that starts at distance $\delta$ from $\overline{\gamma_s}$ ( as measured at a distance $O(1)$ away from the fold $F$).
\item An appropriate transversality hypotheses is satisfied.
\end{enumerate}
Then there exists a stable MMO with signature $1^{k+1}$.
\end{theorem}

\begin{theorem}[Stable MMOs with signature $1^i$][\citealp{MMO}]
Suppose system (\ref{somegenericthreedim}) satisfies assumptions 1. - 4. of Theorem \ref{MMOsigk1} and, the following additional assumption:
\begin{itemize}
\item For $\delta = 0$, the global return point is on the singular strong canard $\overline{\gamma_s}$ and as $\delta$ passes through zero the return point crosses $\overline{\gamma_s}$ with nonzero speed.
\end{itemize}
Suppose now that $\delta= O(\epsilon ^{(1-\mu)/2})>0$. Then, for sufficiently small $0 < \epsilon \ll 1$ and $k \in \mathbf{N}$ such that $2k+1 < \mu^{-1} < 2k+ 3$, the following holds.
For each $i, 1 \leq i \leq k$, there exist subsectors $\overline{I}_i \subset I_i$ with the corresponding distance intervals $(\delta_i^-, \delta_i^+)$ of widths $O(\epsilon^{(1-\mu)/2})$, which have the property that if $\delta \in (\delta_i^-, \delta_i^+)$, then there exists a stable MMO with signature $1^i$.
\end{theorem}





















































\subsection{Singular Hopf Bifurcation}
In this section the folded saddle-node of type 2 and the saddle focus are considered for analysis.
The folded saddle-node o type 2 occurs, when the parameters of the system coinside in such a way that an equilibrium of the full system and a fold point coinside. A saddle-node of type one refers to the case when only an equilibrium of the reduced system crosses a fold, without coinciding with a global equilibrium.
If a saddle-node type 2 occurs for a specific parameter (also plural...), then a singular hopf bifurcation arises at $O(\epsilon)$ away from the equilibrium.
The equilibrium is focus if the eigenvalues corresponding to it are complex and a node if the eigenvalues are real.

\begin{definition}{\textbf{Singular Hopf Bifurcation}}[\citealp{strogatz2007nonlinear}](but also MMO) \\
A singular hopf bifurcation occurs at a certain parameter regime in the system which is $O(\epsilon)$ away from a saddle-node of type 2. There, the eigenvalues of the system cross the imaginary axis, therefore they have a zero real part. Then small oscillations, called limit cycles occur in the system. There are two types of singular Hopf Bifurcation.
The supercritical Hopf Bifurcation occurs when a stable limit cycle arises from an unstable equilibrium point, while the subcritical Hopf Bifurcation causes unstable limit cycles to appear around a stable equilibrium.
\end{definition}

These different orbits caused by a singular Hopf Bifurcation are of interest, because they are SAOs of the fast-slow system in question. Therefore, in this chapter we will give an overview of the different SAOs arising from singular Hopf Bifurcations in different parameter regimes.
The starting point of the analysis is the normal form considered for the folded node in section +++toms section+++, which is then modified to a system that displays a singular Hopf Bifurcation and later on a system with a global return mechanism will be derived.
The first transformation is achieved by adding higher-order terms to the $z$ equation of system (++ toms normal form+++). It then becomes
\begin{align*}
\epsilon \dot{x} &= y - x^2, \\
\dot{y} &= z - x \\
\dot{z} &= - \nu -ax - by - cz,
\end{align*}
which is the normal form for a singular Hopf Bifurcation.
We then consider a coordinate transformation and time rescaling of the form
\begin{align*}
x = \epsilon^{1/2}\overline{x}, \ \ \ y= \epsilon \overline{y},  \ \ \ z = \epsilon^{1/2} \overline{z},\ \ \  t= \epsilon^{1/2} \overline{t}.
\end{align*}
Then the system becomes
\begin{align} \label{sysepsilonenvir}
\overline{x}' &= \overline{y} - \overline{x}^2, \\
\overline{y}' &= \overline{z} - \overline{x}, \\
\overline{z}' &= - \nu - \epsilon^{1/2} a \overline{x} - \epsilon b \overline{y} - \epsilon^{1/2} c \overline{z}.
\end{align}
This transformation can be seen, somewhat equivalently to Section \ref{sec:transform blowup}, as a consideration of a small neighbourhood of the singular point.
As described in Section \ref{sec: threedimfolds}, folded singularity is found by examining the critical manifold $C= \{ (x,y,x) : f:=y-x^2 =0 \}$. The conditions (\ref{eq: non-degeneracy 3d system}) are easily checked and satisfy:
\begin{align*}
f(p_*, \nu, \epsilon)&= y- x^2 =0 \\
\Rightarrow y &= x^2\\
\pd{}{x}f(p_*,\lambda,0) &= -2x = 0,\\
\Rightarrow x &= 0\\
\Rightarrow y &=0\\
\pd{^2}{x^2}f(p_*,\lambda,0) &= -2 \neq 0,\\
D_{(y,z)}f(p_*,\lambda,0) &= (1,0)
\end{align*}
+++++++++++++++++++Help!! Fold conditions do not work out....+ also no idea what the parameters are $\nu, \epsilon$? is it going to zero....+++++++++++++++
The folded singularity is found at $p_*= (0,0,z)$, which makes the further analysis slightly more straightforward.
The equilibria of the system are, such that $p_0= (x,x^2,x)$, where $x$ satisfies:
\begin{align}\label{MMOxsol1}
x = -\frac{1}{2 \epsilon^{1/2} b} \left( (a+c) \pm \sqrt{ (a+c)^2 - 4 b \nu } \right),
\end{align}
and therefore there are two equilibria++++is it correct that i have 2??+++++ at
\begin{align*}
&x_1=-\frac{a+c}{ \epsilon^{1/2} b} + \frac{\nu}{\epsilon^{1/2} (a+c)} + \frac{b \nu^2}{\epsilon^{1/2} (a+c)^3} + ... \\
&x_2= \frac{\nu}{\epsilon^{1/2} (a+c)} + \frac{b \nu^2}{\epsilon^{1/2} (a+c)^3} + ...,
\end{align*}
where a MacLaurin expansion for $\sqrt{ (a+c)^2 - 4 b \nu }$ has been used.
There exists a value for $x$ depending on the parameters $a,b,c$ and $\nu$, where a fold point intersects with the equilibrium. This is at $x_1=0$ and $x_2=0$.
Then, setting (\ref{MMOxsol1}) to zero results in
\begin{align*}
x=-\frac{1}{2 \epsilon^{1/2} b} \left( (a+c) \pm \sqrt{ (a+c)^2 - 4 b \nu } \right)=0\\
\Rightarrow \nu = - \frac{ (a+c)^2 - (a+c)}{4b}.
\end{align*}
Therefore, the location of the singular equilibrium, depends on the parameter values for $a,b,c$.

Since $a, b$ and $c$ are all multiplied by a factor of $\epsilon^{1/2}$ or $\epsilon$ in system (\ref{sysepsilonenvir}), we need $\nu$ to be of $O(\epsilon^{1/2})$ or smaller in order to observe a singular hopf bifurcation.
If $\nu=O(1)$, then the factors of $\epsilon$ in system (\ref{sysepsilonenvir}) do not really contribute to the system and are merely a pertubation of the normal form (++++toms normal form++++).
If $\nu \leq O(\epsilon^{1/2})$, then a singular hopf bifurcation occurs at a distance $\nu =O(\epsilon)$ in parameter space away from the equilibrium.
The eigenvalues of the system (\ref{sysepsilonenvir}) can be found by considering the following Jacobian matrix associated to it:
\begin{equation}
J=\begin{bmatrix}
2x & 1 & 0 \\
-1 & 0 & 1 \\
-\epsilon^{1/2} a & - \epsilon b & - \epsilon^{1/2} c\\
\end{bmatrix}.
\end{equation}
Using a computer package, such as Maple, to solve for the eigenvalues confirms that there exist two complex eigenvalues for the equilibrium where $x=0$.
Since the eigenvalues of the system are complex, the equilibrium is a saddle-focus, which has not been discussed in the analysis of canard trajectories. (+++++loop back to 3dim singularities and why we dont have canards)+++++++
The research of the dynamics, and specifically MMOs, close to a singular Hopf Bifurcation is still ongoing. Here we only consider a few specific cases, where $\nu$ is treated as the main parameter of interest. Furthermore, since the critical manifold in system (\ref{sysepsilonenvir}) is in the shape of a quadratic function, by the geometrical nature of the problem, there is no global return mechanism for the system. Trajectories that leave the close proximity of the equilibrium do not return. In order to get MMOs, additionally to the SAOs a global return mechanism is needed.
This is achieved by modifying system (\ref{sysepsilonenvir}) by adding a cubic term to the $x$ equation. This will change the shape of the critical manifold to an S shaped curve and therefore allow for a global return mechanism. The new system is then the following:
\begin{align*}
\epsilon \dot{x} &= y - x^2 - x^3, \\
\dot{y} &= z - x, \\
\dot{z} &= -\nu -ax -by -cz.
\end{align*}

The expected behaviour of the new system is now to display several SAOs close to the equilibrium, before completing a large amplitude oscillation. This LAO is necessarily of the form of a relaxation oscillation, because there is only one fast variable present in the system. This represents a constraint since the fast subsystem is one dimensional and therefore trajectories are restricted to be monotonic.

There are now many different types of MMOs present, depending on the parameter regimes.
One example is that for small values of $\nu$, where $\nu=O(\epsilon)$, a stable periodic orbit $\Gamma$ arises from the saddle-focus equilibrium. This orbit is tracing out SAOs close tho the repelling sheet of the critical manifold, before completing a relaxation oscillation and returning to its starting point.
However, other bifurcations can occur for these periodic orbits for different parameter regimes. These could be of the form of torus bifurcations or period-doubling. Then there is a possibilities of chaotic MMOs exisiting for these parameters. For decreasing values of $\nu$ here, which is already $O(\epsilon)$, large amplitudes are getting smaller until the system only displays chaotic SAOs(++++++not sure if terminology works like this...+++)

Now it is of interest to consider specific parameter regimes for which the SAOs are constrained to the unstable manifold $W^u(p_*)$, which corresponds to the phase space surrounding the equilibrium $p_*$, while being backward asymptotic to it.
For a supercritical Hopf Bifurcation we just observe the stable oscillation, as before. However, there is another type of bifurcation (+++++WHY+++++++) under certain conditions ($W^u$ tangent tp S)
