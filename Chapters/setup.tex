We wish to study the behaviour of the van der Pol oscillator using blow up. The van der Pol oscillator is a well-studied second order ODE used to model a variety of physical and biological phenomena. A position coordinate \(x(t)\) evolves according to the following equation. 
\begin{equation} \label{eq:vdP}\ddot{x}(t)-\mu\left(1-x^2(t)\right)\dot{x}(t)+x(t)=0   \end{equation}
Here, \(\mu \gg 1\) is a scalar constant. \par 

Currently, it doesn't resemble a anything like what we've seen in \cite{krupa2001}. To make it resemble a fast/slow system, we introduce a new variable. Let \(w=\dot{x}+\mu F(x)\) where \(F(x)=\frac{x^3}{3}-x\). Why have we chosen this function \(F\)? Notice that \(F'(x)=-(1-x^2)\), our nonlinear term in Equation \ref{eq:vdP}. Differentiating \(w\) we obtain
\begin{align*}
    \dot{w}&=\ddot{x}+\mu\od{}{x}\left(\frac{x^3}{3}-x\right)\od{x}{t}\\
    & =\ddot{x}+\mu(x^2-1)\dot{x}\\
    &= -x
\end{align*}
Here, the last equality follows from the van der Pol equation. We now have a two dimensional system.
\[\begin{cases} \dot{x}=w-\mu F(x)\\
 \dot{w}=-x\end{cases}\]
 Let \(y=\frac{w}{\mu}\). Then
 \[\begin{cases} \dot{x}=\mu\left(y-F(x)\right)\\
 \dot{y}=-\frac{x}{\mu}\end{cases}\]
We will pause here although it doesn't look quite right and do a phase plane analysis to better understand the behaviour of the system. Setting each equation equal to zero in turn gives nullclines of \(x=0\) and \(y=F(x)\). 

+++ PHASE PLANE ANALYSIS +++

+++ New transformation to fast/slow system +++


Fast System:
\begin{equation}\label{fastsystem}
    \begin{cases} x'=y-F(x)\\
    y'=-\epsilon x
    \end{cases}
\end{equation}


Slow system:
\begin{equation}\label{slowsystem}
    \begin{cases} \epsilon \dot{x}=y-F(x)\\
    \dot{y}=-x
    \end{cases}
\end{equation}

\subsection{Fold Points}\label{Fold Points}
The fold points are $(x_0^+,y_0^+)=(1,-\dfrac{2}{3})$ and $(x_0^-,y_0^-)=(-1,\dfrac{2}{3})$
\begin{figure}[h]
    \centering
    %\includegraphics{}
    \caption{Our Manifold $f(x,y,\epsilon)$}
    \label{fig:my_label}
\end{figure}

\subsection{Non-degeneracy}
Now that we have established our fast-slow systems (Equation \ref{fastsystem} and \ref{slowsystem}) we need to check our non-degeneration conditions \citep{krupa2001}. 
%Before proceeding it is prudent to note that the nondegeneracy conditions will be opposite for the case where $(x_0,y_0)=(x_0^+,y_0^+)$ to satisfy our system.
Now we first check that $ \pd{^2f}{x^2}(x_0,y_0,0)\neq 0$ then we have,
\begin{equation}
    \pd{^2}{x^2}(y-\dfrac{x^3}{3}-x)=-2x,
\end{equation}
which we can evaluate at our fold points (Section \ref{Fold Points}) to give,
\begin{equation} 
    \begin{cases}
            &\pd[2]{f}{x}(x_0^+,y_0^+,0)=2<0\\
            &\pd[2]{f}{x}(x_0^-,y_0^-,0)=2>0.
    \end{cases}
\end{equation}
Then we can consider that $\pd{f}{y}(x_0,y_0)\neq 0$. We show this by the following,
\begin{align}
        &\pd{f}{y}(x_0^+,y_0^+,0)=1\\
        &\pd{f}{y}(x_0^-,y_0^-,0)=1.
\end{align}
Lastly we need to consider that $g(x_0,y_0,0)\neq 0$. This is easily seen as we find that $g(x_0,y_0,0)=\pm 1$ for our two fold points. From here we are can now consider our transformation.