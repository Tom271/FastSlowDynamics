During this section we will be considering a canard point. This is when our fold point is shifted along the manifold - Figure \ref{fig: Canard Point}. 
\begin{figure}[h!]
    \centering
    \includegraphics[height=5cm,width=8cm]{Canard_Point.png}
    \caption{The reduced flow of our system for a) $\lambda=0$ and b) $\lambda>0$.}
    \label{fig: Canard Point}
\end{figure}
To adequately explain the effect that the canard point will have on our system we will need to consider our system,
\begin{equation}
     \begin{aligned}
        &x'=-y+x^2-\dfrac{x^3}{3},\\
        &y'=\epsilon(x-1),\\
        &\epsilon'=0.\\
        % &\lambda'=0,\\
     \end{aligned}
    \tag{\ref{eq: Fast System}}
\end{equation}
Now we need to consider Equation \ref{eq: Fast System} in terms our our cananrd system. To do this we rewrite our system with an extra parameter $\lambda$, where $\lambda$ is our perturbation of our fold point \citep{krupa2001}. \citet{krupa2001} discusses generally how we should continue with computing our canard system. If we apply his theory to the \vdp system we find,
\begin{equation}
     \begin{aligned}
        &x'=-y+x^2-\dfrac{x^3}{3},\\
        &y'=\epsilon(x-\lambda),\\
        &\epsilon'=0,\\
         &\lambda'=0,\\
     \end{aligned}
        \label{eq: canard system}
\end{equation}
where the change in $\epsilon$ and $\lambda$ are constant. Now, for the remainder of the section, we follow the method of \citet{krupa2001} for the canard system. If we start by rewriting our canard system into the canonical forms we find,
\begin{align}
      &x'=-yh_1(x,y,\epsilon,\lambda)+x^2h_2(x,y,\epsilon,\lambda),\\
        &y'=\epsilon(xh_4(x,y,\epsilon,\lambda)-\lambda h_6(x,y,\epsilon,\lambda)),\\
\end{align}
Where we note that $h_j(x,y,\epsilon,\lambda)=1+O(x,y,\epsilon,\lambda)$ for $j=1,2,4,5$ and $h_3(x,y,\epsilon,\lambda)=O(x,y,\epsilon,\lambda)$. However, we should note that for the \vdp system our only term that is not solely of leading order is $h_2(x,y,\epsilon,\lambda)=1-\frac{x}{3}$. Now we are able to choose such a $\lambda>0$ that produces an equilibrium on our repelling branch $S_r$ for the reduced flow. By doing this we are then able to define the following conditions for our reduced flow on $h_j$,
\begin{align}
    &a_3=\pd{}{x}h_2(0,0,0,0)=-\frac{1}{3},\\
    &A=-a_2+3a_3-(2a_4+2a_5)=-1,
\end{align}
where we notice that our other solutions for $a_i=0$ for $i=1,2,4,5$ are trivial. The reason that we consider the constant $A$ is because we will find that this constant is crucial in our canard point analysis iff $A\neq 0$ \citep{krupa2001}. Following this \citep{krupa2001} discusses the existence of a critical value for $\lambda$ (denoted $\lambda_c$), where our two branches $S_r$ and $S_a$ must connect in a smooth fashion. Now from \textit{Theorem 3.1} we nkow that we must have a transition map at our critical point,
\begin{equation}
    \lambda_c(\sqrt{\epsilon})=-\epsilon(\frac{a_1+a_5}{2}+\frac{A}{8})+O(\epsilon^\frac{3}{2}),
\end{equation}
which can be written as $\lambda_c(\sqrt{\epsilon})=\frac{\epsilon}{8}+O(\epsilon^\frac{3}{2})$ for the \vdp system \citep{krupa2001}. Consider Canard cycles and center manifolds / Freddy Dumortier, Robert Roussarie. for more details on canards in \vdp.


\subsection{Canard Blow-up}
Now similarly to Section \ref{sec: VDP Blowup} we consider various transformations of our coordinate system to be able to be able to consider the non-hyperbolic equilibrium induced by our canard point. However, as we would expect with our new system we should consider a new set of transformations \citep{krupa2001}.
\begin{equation}
    x=\bar{r}\bar{x}, \ y=\bar{r}^2y, \ \epsilon=\bar{r}^2\bar{\epsilon}, \ \lambda=\bar{r}\bar{\lambda}
\end{equation}
Now that we have established the transformation we can then define our transformations for $K_1$ and $K_2$ but it is not necessary to consider the third chart ($K_3$). This is because we find that the attracting slow manifold connects to the repelling slow manifold. As a result of this we find that our flow will `bend back' from $K_2$ into $K_1$ instead of flowing out into the fast flow, which is described by $K_3$. This concept can be described by the Figure \ref{fig: flow in canard}.
\begin{figure}[h!]
    \centering
%    \includegraphics{}
    \caption{Figure describing canard flow in manifold}
    \label{fig: flow in canard}
\end{figure}


Since we have established why we need only consider two charts we can our transformations,
\begin{subequations}
    \begin{align}
        &x=r_1x_1, \ y=r_1^2, \ \epsilon=r_1^2\epsilon_1, \ \lambda=r_1\lambda_1 \label{eq: coordiante K_1}\\ 
        &x=r_2x_2, \ y=r_2^2y_2, \ \epsilon=r^2_2, \ \lambda=r_2\lambda_2 \label{eq: coordinate K_2}
    \end{align}
\end{subequations}
Since these transformations have been defined we should consider our charts. We will first consider chart 2, for analogous reasoning to Section \ref{sec: VDP K_2}. 

\subsubsection{Dynamics in \texorpdfstring{$K_2$}{K2}}
We start by noting that we are considering our invariant plane at $r_2=0$ which will significantly simplify our system for $K_2$. Further we should note that we are taking a transformation in time, $\od{r}{t_2}=\od{t}{t_2}\od{r}{t}=\frac{1}{r_2}\od{r_2}{t}$, as well as in our coordinates. Then if we substitute our time transformation and Equation  \ref{eq: coordinate K_2} into our system of Equations \ref{eq: canard system} we find, 
\begin{subequations}
    \begin{align}
    r_2^2x_2' - r_2x_2r_2'&=-r_2^2y_2h_1+r^2_2x^2_2h_2,\notag\\
    \implies x_2&=-y_2+x_2^2-r_2G_2(x_2,y_2),\\
%     \end{aligned}
% \end{equation*}
% \begin{equation}
%     \begin{aligned}
        r^3_2y_2'-3r_2^2y_2r_2'&=r^2_2(r_2x_2h_4-r_2\lambda_2h_5),\notag\\
        \implies y_2'&=x_2-\lambda_2+r_2G_2(x_2,y_2), \label{eq: K_2 y trans}
    \end{align}
\end{subequations}
where we note that $h_j=h_j(x,y,\epsilon,\lambda)$ for $j=1,2,3,4,5$. We should also recall that $r_2'=\lambda_2'=0$. Notice that we have included an additional term in Equation \ref{eq: K_2 y trans} - we define $G_2(x_2,y_2)$ in the following way, $G(x_2,y_2)=(G_1(x_1,y_1),G_2(x_2,y_2))^T=(-\frac{x^2_2}{3},0)^T$. The reason we also define this vector is to aide in the Melnikov computations which we will see later. \citet{krupa2001} discusses that for this chart we have an interesting result. They note that at $r_2=\lambda_2=0$ our system is integrable which allows us to define a constant of motion $H(x_2,y_2)=\frac{1}{2}\exp{(-2y_2)}\left(y_2-x^2_2+\frac{1}{2}\right)$ which we can easily verify \citep{krupa2001} using the following equations,
\begin{align*}
    x'_2&=e^{2y_2}\pd{H}{y_2}(x_2,y_2),\\
    y_2'&=-e^{2y_2}\pd{H}{x_2}(x_2,y_2).
\end{align*}
\textbf{Further to this we can see, when we consider our reduced system, that we have an equilibrium at the origin, implying that $H(x_2,y_2)=h$. This then allows us to define a trajectory for the orbit by\\
WHY??????????}
\begin{equation}
    \gamma_{c,2}(t_2)=(x_{c,2}(t_2),y_{c,2}(t_2))=\left(\frac{t_2}{2},\frac{t^2_2}{4}-\frac{1}{2}\right)    
\end{equation}
Next we will be continuing our analysis onto $K_1$.


\subsection{Dynamics in \texorpdfstring{$K_1$}{K1}}
For $K_1$ we follow a similar approach to the above. We will use the transformations, 
\begin{equation}
         x=r_1x_1, \ y=r_1^2, \ \epsilon=r_1^2\epsilon_1, \ \lambda=r_1\lambda_1 \tag{\ref{eq: coordiante K_1}},
\end{equation}
to find the relevant pathways of our flows. Now if we first consider the $r_1$ component, 
\begin{align}
    2r_1^2r_1'=r_1^2\epsilon(r_1x_1-r_1\lambda_1), \label{canard: r_1}
\end{align}
where we can call $F=F(x,y,\epsilon,\lambda)=x_1-\lambda_1+O(r_1(r_1+\lambda_1)$. Now we will see the motivation with starting with $y=r_1$ when we transform our other coordinates. Now if we consider $x=r_1x_1$,
\begin{align*}
    r_1r_1'x_1+r_1^2x_1'&=-r_1^2+r_1^2x_1^2,\\
    x_1'&=-1+x_1^2-\frac{x_1r_1'}{r_1},
\end{align*}
where we can use Equation \ref{canard: r_1} to simplify this further - Equation \ref{eq: canard x_1}.
\begin{align}
    x_1'=-1+x_1^2-\frac{x_1}{r_1}\left(\frac{r_1\epsilon_1F}{2}\right) \label{eq: canard x_1}
\end{align}
We now consider our $\epsilon=\epsilon_1r_1^2$ and noting $\epsilon'=0$. Then we have, $r_1^3\epsilon'=-2r_1^2\epsilon_1r_1'$, where we can use Equation \ref{canard: r_1} to simplify to,
\begin{align}
    \epsilon'=-\epsilon_1^2F. \label{canard: epsilon k_1}
\end{align}
Our last transformation is for our new coordinate $\lambda=r_1\lambda$, noting that $\lambda'=0$. Similarly to the above we find $r_1^2\lambda_1'+r_1\lambda_1r_1'=0$ then, 
\begin{equation}
    \lambda'_1=-\frac{\lambda_1\epsilon_1F}{2}, 
\end{equation}
which is a trivial rearrangement as seen in Equation \ref{canard: epsilon k_1}. Now if we combine the above we find that our transformed system is of the following form,
\begin{subequations}
    \begin{align}
            r_1'&=\frac{\epsilon}{2}(r_1x_1-r_1\lambda_1), \\
            % \label{canard: r_1}
            x_1'&=-1+x_1^2-\frac{x_1\epsilon_1F}{2},\\
            \epsilon'&=-\epsilon_1^2F,\\
            \lambda'_1&=-\frac{\lambda_1\epsilon_1F}{2}.
    \end{align}
    \label{canard: system of equations}
\end{subequations}
% Unique becuase of exponetial attraction in canard case - note it is the reversal of figure 2.4 for uniqueness
From this system we are now able to make some deductions. We first can observe that the hyperplanes are along the $r_1=\epsilon_1=\lambda_1=0$ with an invariant line at $l_1=\{(x_1,0,0,0): x_1\in\Re\}$ \citep{krupa2001}. As \citet{krupa2001} discusses the equilibria present at the end of both of our branches - Figure \ref{fig: Canard Point} - which are found at $p_a=(-1,0,0,0) \ \text{and} \ p_r=(1,0,0,0)$ \citep{krupa2001}. Now we can go one step further, we can consider Equation \ref{canard: system of equations} and find the eigenvalues of the system for the invariant planes. We find that, 
\begin{equation}
    J-\lambda I= \begin{bmatrix}
    2x-\lambda & 0 & 0 & 0  \\
    0 & -\lambda & 0 & 0&\\
    0 & 0 & -\lambda & 0 \\
    0 & 0 & 0 & -\lambda
\end{bmatrix},
\end{equation}
which clearly has three zero eigenvalues and one non-zero eigenvalue $\lambda=\pm 2$. Which further empahsises that our equilibrium point is non-hyperbolic.  
% \begin{align}
%     \label{canard: x_1}
% \end{align}


% This effect can cause numerous effects within applications of dynamical systems. 

