\begin{figure}[h!]
	\centering
	\includegraphics[height=5cm,width=8cm]{Canard_Point.png}
	\caption{The reduced flow where a) $\lambda=0$ and b) $\lambda>0$.}
	\label{fig: Canard Point}
\end{figure}



Considering the Van der Pol System as before:
\begin{equation}
\begin{aligned}
&x'=-y+x^2-\dfrac{x^3}{3},\\
&y'=\epsilon(x-1),\\
\end{aligned}
\tag{\ref{eq: Fast System}}
\end{equation}
we notice that the equilibrium of the system depends on the two nullclines $ x'=0$ and $y'=0$. These are in the shape of a cubic funcion and in the shape of a vertical line at $x=1$. 
The idea in this section is to replace the nullcline $x=1$ by $x = \lambda$. This can be seen as shifting the equilibrium of the system along the critical manifold $S$ by varying the parameter $\lambda$.
This gives rise to a generalised Van der Pol system:
\begin{equation}
\begin{aligned}
&x'=-y+x^2-\dfrac{x^3}{3},\\
&y'=\epsilon(x-\lambda).\\
\end{aligned}
\label{eq: canard system}
\end{equation}
In this section, the dynamics in system \ref{eq: canard system} is analysed. In order to do so, we need the definition of a canard.
\begin{definition}{\textbf{ Canard}}[\citealp{Kuehn}]
A trajectory of a fast-slow system is called a canard if it stays within $O(\epsilon)$ close to the repelling branch $S^r$ of the slow manifold $S$, for some time of $O(1)$ on the slow time scale $\tau = \epsilon t$.
\end{definition}
Furthermore, the following definition turns out to be useful as well:
\begin{definition}{\textbf{Maximal Canard}}[\citealp{Kuehn}]
If the equilibrium coincides with a fold point of a fast-slow system, that is, at the intersection of $S^a$ and $S^r$,the trajectory passing through this point is called a maximal canard. 
\end{definition}
The intuition of the canard problem close to a fold point is given in Figure \ref{fig: Canard Point}.

Equivalently to the analysis of the fold point in Section \ref{sec:the-van-der-pol-equation}, some nondegeneracy conditions are defined.
These are, as before, applied at the fold point $(0,0)$. Note that in contrast to the nondegeneracy conditions in (\ref{NonDeg}), the transversality condition $g(0,0,0) \neq 0$ is not satisfied. Therefore higher order conditions on $g$ have to be employed, in particular these are nonzero derivatives of $g$ with respect to $x$ and $\lambda$.
The nondegeneracy and transversality conditions for the canard case are:
\begin{equation}
f(0,0,0,0)=0, \ \pd{}{x}f(0,0,0,0)=0, \ g(0,0,0,0)=0, \label{eq: canard sing condition}
\end{equation}
\begin{equation}
\begin{aligned}
&\pd{^2}{x^2}f(0,0,0,0)\neq 0, \ \pd{}{y}f(0,0,0,0)\neq 0,\\
&\pd{}{x}g(0,0,0,0)\neq 0, \ \pd{}{\lambda}g(0,0,0,0)\neq 0. \label{eq: canard non-d condition}
\end{aligned}
\end{equation}
 \citep{krupa2001}. 


Now that the conditions have been defined we can continue onto the \vdp system,
\begin{equation}
\begin{aligned}
&x'=-y+x^2-\dfrac{x^3}{3},\\
&y'=\epsilon(x-1),\\
&\epsilon'=0.\\
% &\lambda'=0,\\
\end{aligned}
\tag{\ref{eq: Fast System}}
\end{equation}
Now we need to consider Equation \ref{eq: Fast System} in terms our our cananrd system. To do this we rewrite our system with an extra parameter $\lambda$, where $\lambda$ is our perturbation of our fold point \citep{krupa2001}. \citet{krupa2001} discusses generally how we should continue with computing our canard system. If we apply his theory to the \vdp system we find,
\begin{equation}
\begin{aligned}
&x'=-y+x^2-\dfrac{x^3}{3},\\
&y'=\epsilon(x-\lambda),\\
&\epsilon'=0,\\
&\lambda'=0,\\
\end{aligned}
\label{eq: canard system}
\end{equation}
where the change in $\epsilon$ and $\lambda$ are constant. Now, for the remainder of the section, we follow the method of \citet{krupa2001} for the canard system. If we start by rewriting our canard system into the canonical forms we find,
\begin{subequations}
	\begin{align}
		&x'=-yh_1(x,y,\epsilon,\lambda)+x^2h_2(x,y,\epsilon,\lambda),\\
		&y'=\epsilon(xh_4(x,y,\epsilon,\lambda)-\lambda h_6(x,y,\epsilon,\lambda)),
	\end{align}
\end{subequations}
Where we note that $h_j(x,y,\epsilon,\lambda)=1+O(x,y,\epsilon,\lambda)$ for $j=1,2,4,5$ and $h_3(x,y,\epsilon,\lambda)=O(x,y,\epsilon,\lambda)$. However, we should note that for the \vdp system our only term that is not solely of leading order is $h_2(x,y,\epsilon,\lambda)=1-\frac{x}{3}$. Now we are able to choose such a $\lambda>0$ that produces an equilibrium on our repelling branch $S_r$ for the reduced flow. By doing this we are then able to define the following conditions for our reduced flow on $h_j$,
\begin{subequations}
		\begin{align}
		&a_3=\pd{}{x}h_2(0,0,0,0)=-\frac{1}{3},\\
		&A=-a_2+3a_3-(2a_4+2a_5)=-1,
		\end{align}
\end{subequations}
where we notice that our other solutions for $a_i=0$ for $i=1,2,4,5$ are trivial. The reason that we consider the constant $A$ is because we will find that this constant is crucial in our canard point analysis iff $A\neq 0$ \citep{krupa2001}. Following this \citet{krupa2001} discuss the existence of a critical value for $\lambda$ (denoted $\lambda_c$), where our two branches $S_r$ and $S_a$ must connect in a smooth fashion. Now from Theorem \ref{Theorem 3.1} we know that we must have a transition map at our critical point,
\begin{equation}
\lambda_c(\sqrt{\epsilon})=-\epsilon\left(\frac{a_1+a_5}{2}+\frac{A}{8}\right)+O(\epsilon^\frac{3}{2}),
\end{equation}
which can be written as $\lambda_c(\sqrt{\epsilon})=\frac{\epsilon}{8}+O(\epsilon^\frac{3}{2})$ for the \vdp system \citep{krupa2001}. Consider Canard cycles and center manifolds / Freddy Dumortier, Robert Roussarie. for more details on canards in \vdp.


\subsection{Canard Blow-up}
Now similarly to Section \ref{sec: VDP Blowup} we consider various transformations of our coordinate system to be able to consider the non-hyperbolic equilibrium induced by our canard point. However, as we would expect with our new system we should consider a new set of transformations \citep{krupa2001}.
\begin{equation}
x=\bar{r}\bar{x}, \ y=\bar{r}^2y, \ \epsilon=\bar{r}^2\bar{\epsilon}, \ \lambda=\bar{r}\bar{\lambda}
\end{equation}
Now that we have established these transformation we can then define our transformations for $K_1$ and $K_2$ but it is not necessary to consider the third chart ($K_3$). This is because we find that the attracting slow manifold connects to the repelling slow manifold. As a result of this we find that our flow will `bend back' from $K_2$ into $K_1$ instead of flowing out into the fast flow, which is described by $K_3$. This concept can be described by the Figure \ref{fig: flow in canard},
\begin{figure}[h!]
	\centering
\includegraphics[width=8cm, height=6cm]{Images/vdPhopf-Moment-bendback}
	\caption{The \vdp system for the canard case.}
	\label{fig: flow in canard}
\end{figure}\newpage
where we can clearly see that our flow need not enter chart $ K_3 $ as it `bends back' on itself in the chart $ K_2 $ - see Section \ref{sec:matlab-stuff} for a discussion on why $ \epsilon\neq0 $. Since we have established why we need only consider two charts we can our transformations,
\begin{subequations}
	\begin{align}
	&x=r_1x_1, \ y=r_1^2, \ \epsilon=r_1^2\epsilon_1, \ \lambda=r_1\lambda_1 \label{eq: coordiante K1}\\ 
	&x=r_2x_2, \ y=r_2^2y_2, \ \epsilon=r^2_2, \ \lambda=r_2\lambda_2 \label{eq: coordinate K2}
	\end{align}
\end{subequations}
Since these transformations have been defined we should consider our charts. We will first consider chart 2, for analogous reasoning to Section \ref{sec: VDP K_2}. 

\subsubsection{Dynamics in \texorpdfstring{$K_2$}{K2}}
We start by noting that we are considering our invariant plane at $r_2=0$ which will significantly simplify our system for $K_2$. Further we should note that we are taking a transformation in time, $\od{r}{t_2}=\od{t}{t_2}\od{r}{t}=\frac{1}{r_2}\od{r_2}{t}$, as well as in our coordinates. Then if we substitute our time transformation and Equation  \ref{eq: coordinate K2} into our system of Equations \ref{eq: canard system} we find, 
\begin{subequations}
	\begin{align}
	r_2^2x_2' - r_2x_2r_2'&=-r_2^2y_2h_1+r^2_2x^2_2h_2,\notag\\
	\implies x'_2&=-y_2+x_2^2-r_2G_2(x_2,y_2),\\
	%     \end{aligned}
	% \end{equation*}
	% \begin{equation}
	%     \begin{aligned}
	r^3_2y_2'-3r_2^2y_2r_2'&=r^2_2(r_2x_2h_4-r_2\lambda_2h_5),\notag\\
	\implies y_2'&=x_2-\lambda_2+r_2G_2(x_2,y_2), \label{eq: K2 y trans}
	\end{align}
	\label{eq: reduced canard k2}
\end{subequations}
where we note that $h_j=h_j(x,y,\epsilon,\lambda)$ for $j=1,2,3,4,5$. We should also recall that $r_2'=\lambda_2'=0$. Notice that we have included an additional term in Equation \ref{eq: K2 y trans} - we define $G_2(x_2,y_2)$ in the following way, $G(x_2,y_2)=(G_1(x_1,y_1),G_2(x_2,y_2))^T=(-\frac{x^2_2}{3},0)^T$. The reason we also define this vector is to aide in the Melnikov computations which we will see later. \citet{krupa2001} discusses that for this chart we have an interesting result. They note that at $r_2=\lambda_2=0$ our system is integrable which allows us to define a constant of motion $H(x_2,y_2)=\frac{1}{2}\exp{(-2y_2)}\left(y_2-x^2_2+\frac{1}{2}\right)$ which we can easily verify \citep{krupa2001} using the following equations,
\begin{subequations}
	\begin{align}
	x'_2&=e^{2y_2}\pd{H}{y_2}(x_2,y_2),\\
	y_2'&=-e^{2y_2}\pd{H}{x_2}(x_2,y_2).
	\end{align}
\end{subequations}
Further to this we can see, when we consider our reduced system, that we have an equilibrium at the origin, implying that $H(x_2,y_2)=h$.
considering the reduced system (Equation \ref{eq: reduced canard k2}) we find from $ H(x_2,y_2)=0 $ that,
\begin{subequations}
	\begin{align}
	x_2'&=\frac{1}{2}\ \	\implies x_2=\frac{t_2}{2}+A, \label{canard: trajectory x}\\
	y_2'&=\frac{t_2}{2}\ \implies y_2=\frac{t_2^2}{4}-\frac{1}{2}, \label{canard: trajectory y}
	\end{align}
\end{subequations} 
where we have directly integrated Equation \ref{canard: trajectory x} with respect to our time ($ t_2 $). However, we can note that we are able to choose $ A=0 $ as we are considering an autonomous (time-invariant) system. Then for Equation \ref{canard: trajectory y} we are able to rearrange constant of motion at zero to give, $ y_2=x_2^2-\frac{1}{2} $. Clearly from this analysis we are then able to define our trajectories in terms of $ \gamma_{c,2} $, 
\begin{equation}
\gamma_{c,2}(t_2)=(x_{c,2}(t_2),y_{c,2}(t_2))=\left(\frac{t_2}{2},\frac{t^2_2}{4}-\frac{1}{2}\right).   
\end{equation}
Now that we have established that we must have a flow on our second chart, then there must also exist transition maps. Therefore this now enables us to consider the first chart in the following section.


\subsection{Dynamics in \texorpdfstring{$K_1$}{K1}}\label{sec:dynamics-in-texorpdfstringk1k1}
For $K_1$ we follow a similar approach to the above. We will use the transformations, 
\begin{equation}
x=r_1x_1, \ y=r_1^2, \ \epsilon=r_1^2\epsilon_1, \ \lambda=r_1\lambda_1 \tag{\ref{eq: coordiante K1}},
\end{equation}
to find the relevant pathways of our flows. Now if we first consider the $r_1$ component, 
\begin{align}
2r_1^2r_1'=r_1^2\epsilon(r_1x_1-r_1\lambda_1), \label{canard: r1}
\end{align}
where we can call $F=F(x,y,\epsilon,\lambda)=x_1-\lambda_1+O(r_1(r_1+\lambda_1)$. Now we will see the motivation with starting with $y=r_1$ when we transform our other coordinates. Now if we consider $x=r_1x_1$,
\begin{align*}
r_1r_1'x_1+r_1^2x_1'&=-r_1^2+r_1^2x_1^2,\\
x_1'&=-1+x_1^2-\frac{x_1r_1'}{r_1},
\end{align*}
where we can use Equation \ref{canard: r1} to simplify this further - Equation \ref{eq: canard x1}.
\begin{align}
x_1'=-1+x_1^2-\frac{x_1}{r_1}\left(\frac{r_1\epsilon_1F}{2}\right) \label{eq: canard x1}
\end{align}
We now consider our $\epsilon=\epsilon_1r_1^2$ and noting $\epsilon'=0$. Then we have, $r_1^3\epsilon'=-2r_1^2\epsilon_1r_1'$, where we can use Equation \ref{canard: r1} to simplify to,
\begin{align}
\epsilon'=-\epsilon_1^2F. \label{canard: epsilon k1}
\end{align}
Our last transformation is for our new coordinate $\lambda=r_1\lambda$, noting that $\lambda'=0$. Similarly to the above we find $r_1^2\lambda_1'+r_1\lambda_1r_1'=0$ then, 
\begin{equation}
\lambda'_1=-\frac{\lambda_1\epsilon_1F}{2}, 
\end{equation}
which is a trivial rearrangement as seen in Equation \ref{canard: epsilon k1}. Now if we combine the above we find that our transformed system is of the following form,
\begin{subequations}
	\begin{align}
	r_1'&=\frac{\epsilon}{2}(r_1x_1-r_1\lambda_1), \\
	% \label{canard: r1}
	x_1'&=-1+x_1^2-\frac{x_1\epsilon_1F}{2},\\
	\epsilon'&=-\epsilon_1^2F,\\
	\lambda'_1&=-\frac{\lambda_1\epsilon_1F}{2}.
	\end{align}
	\label{canard: system of equations}
\end{subequations}
% Unique becuase of exponetial attraction in canard case - note it is the reversal of figure 2.4 for uniqueness
From this system we are now able to make some deductions. We first can observe that the hyperplanes are along the $r_1=\epsilon_1=\lambda_1=0$ with an invariant line at $l_1=\{(x_1,0,0,0): x_1\in\Re\}$ \citep{krupa2001}. As \citet{krupa2001} discusses the equilibria present at the end of both of our branches - Figure \ref{fig: Canard Point} - which are found at $p_a=(-1,0,0,0) \ \text{and} \ p_r=(1,0,0,0)$ \citep{krupa2001}. Now we can go one step further, we can consider Equation \ref{canard: system of equations} and find the eigenvalues of the system for the invariant planes. We find that, 
\begin{equation}
J-\lambda I= \begin{bmatrix}
2x-\lambda & 0 & 0 & 0  \\
0 & -\lambda & 0 & 0&\\
0 & 0 & -\lambda & 0 \\
0 & 0 & 0 & -\lambda
\end{bmatrix},
\end{equation}
which clearly has three zero eigenvalues and one non-zero eigenvalue $\lambda=\pm 2$. Which further empahsises that our equilibrium point is non-hyperbolic. As a result we intuitively expect that something interesting occurs at this point. In the section following we will be considering what effect these mappings and eigenvalues will have on our system.


\subsection{Effect of the Canard Point}\label{sec:effect-of-the-canard-point}
Now that we have shown that there must exist a flow around our fold point we should now consider the global effect of the canard point. We can see by considering the  system of Equations \ref{canard: system of equations} that our equilibriums are at $ (x,y)=(\lambda,\lambda^2[\frac{1-\lambda}{3}]) $ and find the eigenvalues from the matrix, 
\begin{equation}
A-\sigma I=\begin{bmatrix}
2x-x^2-\sigma&-1&0&0\\
\epsilon&-\sigma&x-\lambda&-\epsilon\\
0&0&-\sigma&0\\
0&0&0&-\sigma
\end{bmatrix}=\sigma^2(\sigma^2+\sigma(x^2-2x)+\epsilon).
\end{equation}
%\begin{equation}
%A-\sigma I=\begin{bmatrix}
%2\lambda-\lambda^2-\sigma&-1&0&0\\
%\epsilon&-\sigma&0&-\epsilon\\
%0&0&-\sigma&0\\
%0&0&0&-\sigma
%\end{bmatrix}.
%\end{equation}
%\textbf{Then our eigenvalues are, $ \sigma=(2-x)x \ \text{and} \ \sigma=0 $, noting that we have an upper triangular matrix. Then we can note that we have a complex eignevalue which causes a Hopf bifurcation, as shown below:}%wrong maths
From this we are about to find the eigenvalues of the system, $ \sigma=0 $ and $ \sigma=\frac{2x-x^2\pm\sqrt{(x^2-2x)^2-4\epsilon}}{2} $. Then we consider the values at our equilibrium, $ x=\lambda $, to find that we have a Hopf Bifurcation when $ 4\epsilon>(x^2-2x)^2 $ or when $ \lambda=2 \ \text{or} \ 0 $. This then leads to the following trajectories within the flow - Figure \ref{fig: 4 canard }.

\begin{figure}[h!]
	\centering
	\begin{subfigure}[t]{0.45\textwidth}
		\centering
		\includegraphics[width=.8\linewidth]{vdPhopf-Moment-1.jpg}
		\caption{The initial flow within the system.} \label{fig:timing1}
	\end{subfigure}
	\hfill
	\begin{subfigure}[t]{0.45\textwidth}
		\centering
		\includegraphics[width=.8\linewidth]{vdPhopf-Moment-2.jpg}
		\caption{The flow as it hits the slow manifold.} \label{fig:timing2}
	\end{subfigure}
	
	\vspace{1cm}
	\begin{subfigure}[t]{0.45\textwidth}
		\centering
		\includegraphics[width=.8\linewidth]{vdPhopf-Moment-3.jpg}
		\caption{The flow as it intersects with the fold point.} \label{fig:timing3}
	\end{subfigure}
	\hfill
	\begin{subfigure}[t]{0.45\textwidth}\centering
		% just an empty subfigure to shift C below A
		\includegraphics[width=.8\linewidth]{vdPhopf-Moment-4.jpg}
		\caption{The Hopf bifurcation due to the canard point.}\label{fig:timing4}
	\end{subfigure}
	\caption{The trajectories associated with the canards case of the \vdp system.}
	\label{fig: 4 canard }
\end{figure}
From Figure \ref{fig: 4 canard } we can see the progression of our flow over the system. From Figure \ref{fig:timing1} we see that the flow starts at an initial condition of $ (x,y)=(1,1) $ and travels along the fast flow towards the attracting branch. Then from Figure \ref{fig:timing2} the flow has hit the attracting branch, where it then follows along the slow flow towards the fold point at $ (x,y)=(0,0) $, which is described by Figure \ref{fig:timing3}. Then from Figures \ref{fig:timing3} and \ref{fig:timing4} we can observe the Hopf bifurcation. This is because we make note that the canard point is present at $ -\lambda $, which in essence pushes the flow up the repelling branch (see Figure \ref{fig: Canard Point}) until the flow is sufficiently far from the fold point where it will then repel towards the attracting branch, starting the oscillation - Figure \ref{fig:timing4}. Moreover, it is worth noting that our Hopf bifurcation only exists when we are in an arbitrarily small region, $ O(\epsilon) $ \cite{Eckhaus} - this idea is further discussed in Section \ref{sec:singular-hopf-bifurcation}.
%\begin{figure}[h!]\centering
%%	\includegraphics[]{}
%	\caption{Development of the Hopf Bifurcation. \textbf{Tom can you screenshot the flow in 4 places?}}
%	\label{fig: Hopf}
%\end{figure}
%Figure \ref{fig: 4 canard } shows that we have an unstable periodic solution within our canard system. In addition to this we know that our canard system will only exist within a small region of $ O(\epsilon) $ \citep{Eckhaus}. Moreover, we can see from Figure \ref{fig: Hopf} that our flow follows the expected path, as we saw in ++++++++++++++++++++++++\\

\subsubsection{Singular Hopf Bifurcation}\label{sec:singular-hopf-bifurcation}
Furthermore, in the \vdp we are able to find a singular Hopf Bifurcation when $ \lambda=1 $. Then to model this behaviour we need to consider a small perturbation along the slow flow where we will have, from Equation \ref{eq: canard system},%discuss what the solutions for x is in the paper 
\begin{equation}
\dot{y}=\lambda-x+\bar{\nu} y,
\end{equation}
where $ \nu $ is of order $ O(\epsilon) $, thus small. We can immediately see that when $ \bar{\nu}=0 $ that we have our original flow at our equilibrium but we are now able to perturb our flow over a small domain, which are described in Figures \ref{fig: Hopf}. We can also see how our system behaves when our $ \nu $ is of larger order than $ O(\epsilon) $,

\begin{figure}[h]\centering
	\includegraphics[height=6cm,width=10cm]{Images/CanardPointcircle}
	\caption{The flow within our canard system \citep{krupa2001}.}
	\label{fig: canard flow circle}
\end{figure}\newpage
where it is clear that our Hopf bifurcation is the periodic solution in the centre of Figure \ref{fig: canard flow circle} but we can see that below our special flow $ \bar{\gamma_c} $ (for the flow outside of the domain $ O(\epsilon) $),our solution traverses through our equilbrium into our fast flow as we would expect in our original system.


%\subsubsection{Singular Hopf Bifurcation}\textbf{Does this section tie in anywhere else?}
%In this section we will further expand on our Hopf Bifurcation of the previous section (Section \ref{sec:effect-of-the-canard-point}). We note that we get a singular bifurcation iff our system is equivalent to Equation \ref{eq: Fast System}, \st $ \lambda=1 $. \citet{Eckhaus} discusses that our bifuraction will only exist within a small range of $ O(\epsilon) $. Then to model this behaviour we need to consider a small perturbation along the slow flow where we will have, from Equation \ref{eq: canard system},%discuss what the solutions for x is in the paper 
%\begin{equation}
%\dot{y}=\lambda-x+\bar{\nu} y,
%\end{equation}
%where $ \nu $ is of order $ O(\epsilon) $, thus small. We can immediately see that when $ \bar{\nu}=0 $ that we have our orignal flow at our equilbrium but we are now able to perturb our flow over a small domain, which are described in Figures \ref{fig: Hopf}. We can also see how our system behaves when our $ \nu $ is of larger order than $ O(\epsilon) $,
%
%\begin{figure}[h]\centering
%	\includegraphics[height=6cm,width=10cm]{Images/CanardPointcircle}
%	\caption{The flow within our canard system \citep{krupa2001}.}
%	\label{fig: canard flow circle}
%\end{figure}\newpage
%where it is clear that our Hopf bifurcation is the periodic solution in the centre of Figure \ref{fig: canard flow circle} but we can see that below our special flow $ \bar{\gamma_c} $, our solution traverses through our equilbrium into our fast flow as we would expect in our original system.




\subsubsection{Separation of the Manifolds}\label{sec:separation-of-the-manifolds}
\begin{figure}[h!]\centering
	\includegraphics[height=8cm,width=10cm]{Images/Separation}
	\caption{Separation of $ M_a $ and $ M_r $ \citep{Kuehn}.}
	\label{fig: splitting}
\end{figure}
%\textbf{Discuss splitting on the manifold}
Continuing on from the singular Hopf bifurcation we might find that the canard point forces our branches to split. In other words we are looking for when the attracting and repelling branches are no longer connected, as show in Figure \ref{fig: splitting}. To do this we would use Melnikov Computations to show that our manifolds split - see \textit{Extending Geometric Singular Perturbation Theory to Nonhyperbolic Points - Fold and Canard Points in Two Dimensions} \citep{krupa2001} for direct use. To discover whether we have a splitting between our branches we need to consider our $ y $ coordinates \wrt our second chart \st $ y_{a,2}(0)-y_{r,2}(0) $ is a distance function which can be written as $ D_c(r_2,\lambda_2)=H(0,y_{a,2}(0))-H(0,y_{r,2}(0)) $ as we note that $ \pd{}{y_2}H(0,y_2)\neq 0 $ \citep{krupa2001}. From here we can use the following proposition,
\begin{prop}
	[\citealp{krupa2001}]
	For a small enough $ \rho $ and $ \mu $ the distance function has the expansion
	\begin{equation*}
	D_c(r_2,\lambda_2)=d_{r_2}r_2+d_{\lambda_2}\lambda_2+O(2),
	\end{equation*}
	where we have defined,
	\begin{subequations}
		\begin{align}
		d_{r_2}&=\int_{-\infty}^{\infty}\text{grad}H(\gamma_{c,2}(t))\cdot G(\gamma_{c,2}(t))dt,\\
		d_{\lambda_2}&=\int_{-\infty}^{\infty}\text{grad}H(\gamma_{c,2}(t))\cdot (0,-1)^T,
		\end{align}
	\end{subequations}
	and our matrix $ G(\gamma_{c,2}(t)) $ in Section \ref{sec:dynamics-in-texorpdfstringk1k1} with $ \gamma_{c,2} $ as our critical trajectory. 
\end{prop}
Then, following the proof provided by \citet{krupa2001}, we find that we will have a split occurring between our branches if the canard falls outside of our domain of order $ O(e^{-\frac{c}{\epsilon}}) $ \st $ D_c(r_2,\lambda_2)\neq 0 $. As a result of we see a flow similar to Figure \ref{fig: splitting} whereby we find that our flow could either jump off under the fast flow - see Figure \ref{fig: vdp flow diagram e} - or we might find that the flow could be trapped in the canard region and then be repelled back to the attracting manifold, as we see with our connected system - Figure \ref{fig: 4 canard }. 

\textbf{Good to have a figure if possible}



