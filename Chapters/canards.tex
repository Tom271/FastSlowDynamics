\begin{figure}[h!]
	\centering
	\includegraphics[height=5cm,width=8cm]{Canard_Point.png}
	\caption{The reduced flow where a) $\lambda=0$ and b) $\lambda>0$.}
	\label{fig: Canard Point}
\end{figure}

We have now found the global dynamics for the Van der Pol system with $\epsilon >0$ for $\lambda =1$. It is now of interest to investigate whether changing $\lambda$ causes a qualitative change in the dynamics of the Van der Pol system. Consider the Van der Pol System with $\lambda \neq 1$,

\begin{equation}
\begin{aligned}
&x'=-y+x^2-\dfrac{x^3}{3},\\
&y'=\epsilon(x-\lambda).\\
\end{aligned}
\label{eq: canard system}
\end{equation}
Since the nullcline $\dot{y}=0 \Rightarrow x=\lambda$ depends on $\lambda$, varying $\lambda$ corresponds to a shift of the equilibrium along the S shaped curve. Numerical experiments show that solutions to the full system are either relaxation oscillations when the equilibrium lies between the fold points. That is, when it is on the repelling branch. When the equilibrium lies on the attracting branch, all trajectories converge to this point (when $\lambda>2$ or $\lambda<0$). Naturally, there must be a change of stability when $\lambda=0$ and when $\lambda=2$. In this section we aim to explain the nature of this shift in stability and its effect on the global dynamics rigorously. Figure \ref{fig:varLambda} gives an initial idea of the behaviour when $\lambda$ is close to 0. We will restrict attention to this point as the case for $\lambda=2$ is entirely analogous. Trajectories that pass close to this fold point are called canards, more rigorous definitions are given below.\\
 

\begin{figure}[h!]
	\centering
	\begin{subfigure}[t]{0.33\textwidth}
		\centering
		\includegraphics[width=\linewidth]{VDPlambdaneg05}
		\caption{$\lambda=-0.5$}
		\label{}
	\end{subfigure}
	\hfill
		\begin{subfigure}[t]{0.32\textwidth}
		\includegraphics[width=\linewidth]{VDPlambda0}
		\caption{$\lambda=0$}
		\label{} 
	\end{subfigure}
\hfill
	\begin{subfigure}[t]{0.33\textwidth}
		\includegraphics[width=\linewidth]{VDPlambdapos05}
		\caption{$\lambda=0.5$}
		\label{} 
	\end{subfigure}
	\caption{Example trajectories for varying $\lambda$. Note the small amplitude oscillations when $\lambda=0$. }
	\label{fig:varLambda}
\end{figure}

\begin{definition}[\textbf{Canard} \citep{Kuehn}]
A trajectory of a fast-slow system is called a canard if it stays within $O(\epsilon)$ close to the repelling branch $S^r$ of the slow manifold $S$, for some time of $O(1)$ on the slow time scale $\tau = \epsilon t$.
\end{definition}
Furthermore, the following definition turns out to be useful as well:
\begin{definition}[\textbf{Maximal Canard} \citep{Kuehn}] \label{maxcanard}
The trajectory passing through the intersection of $S^a$ and $S^r$ is called a maximal canard. 
\end{definition}
\begin{definition}[\textbf{Singular Canard} \citep{Kuehn}]
	Trajectories of the slow flow lying in the attracting and repelling parts of the critical manifold are called singular canards.
\end{definition}
We expect singular canards in the reduced system to perturb to maximal canards in the full system. We follow a similar process as to when $\lambda=1$. First, some nondegeneracy conditions are required. These are, as before, applied at the fold point $(0,0)$. Note that in contrast to the nondegeneracy conditions in (\ref{NonDeg}), the transversality condition $g(0,0,0) \neq 0$ is not satisfied. Therefore higher order conditions on $g$ have to be employed, in particular these are nonzero derivatives of $g$ with respect to $x$ and $\lambda$. The fact that $g_x(0,0,0) \neq 0$ guarantees the existence of transversal intersection of the two nullclines, which is crucial in order to conclude persistence of the dynamics. The constraint on the derivative of $g$ with respect to $\lambda$ ensures that the nullcline passes through the fold point with nonzero speed \citep{Kuehn}. The nondegeneracy and transversality conditions for the canard case are  \citep{krupa2001},
\begin{equation}
f(0,0,0,0)=0, \ \pd{}{x}f(0,0,0,0)=0, \ g(0,0,0,0)=0, \label{eq: canard sing condition}
\end{equation}
\begin{equation}
\begin{aligned}
&\pd{^2}{x^2}f(0,0,0,0)\neq 0, \ \pd{}{y}f(0,0,0,0)\neq 0,\\
&\pd{}{x}g(0,0,0,0)\neq 0, \ \pd{}{\lambda}g(0,0,0,0)\neq 0. \label{eq: canard non-d condition}
\end{aligned}
\end{equation}
Now that these conditions have been defined we can consider, equivalent to the argument in Section \ref{sec: VDP Blowup}, the extended \vdp system,
\begin{equation}
\begin{aligned}
&x'=-y+x^2-\dfrac{x^3}{3},\\
&y'=\epsilon(x-\lambda),\\
&\epsilon'=0,\\
&\lambda'=0.\\
\end{aligned}
\label{eq: extended canard system}
\end{equation}
Now, for the remainder of the section, we apply the method of \citet{krupa2001} to the Van der Pol System. The canonical form for \ref{eq: canard system} is,
\begin{equation} \label{canardysy2var}
	\begin{aligned}
		x'&=-yh_1(x,y,\epsilon,\lambda)+x^2h_2(x,y,\epsilon,\lambda) + \epsilon h_3(x,y,\lambda,\epsilon)\\
                        &= -y + x^2 \left( 1- \frac{x}{3} \right),\\
		y'&=\epsilon(xh_4(x,y,\epsilon,\lambda)-\lambda h_5(x,y,\epsilon,\lambda) + y h_6(x,y,\lambda,\epsilon)) \\
                        &= \epsilon( x- \lambda).
	\end{aligned}
\end{equation}
It follows that $h_1 = 1$, $h_2 = 1-\frac{x}{3}$, $h_3=0$, $h_4 =1$, $h_5=1$ and $h_6=0$.
To simplify the following computations, we define
\begin{align*}
a_1=\pd{}{x}h_3(0,0,0,0)=& 0, \ \ \
a_2=\pd{}{x}h_1(0,0,0,0)= 0,\ \ \ 
a_3=\pd{}{x}h_2(0,0,0,0)=-\frac{1}{3},\\
a_4&=\pd{}{x}h_4(0,0,0,0)=0,\ \ \
a_5=h_6(0,0,0,0)=0.
\end{align*}
Furthermore, we can define the quantity,
\begin{align*}
A=-a_2+3a_3-(2a_4+2a_5)=-1,
\end{align*}
which is important in the following analysis, in particular for $A \neq 0$ \citep{krupa2001}. 
Similar to the procedure in Section \ref{sec: VDP Blowup}, sections of the dynamical system can be defined, in order to monitor the in- and outgoing trajectories. In this case we are interested in two sections of the neighbourhood $U$, defined as in Section \ref{sec: VDP Blowup}, that monitor $S^a$ and $S^r$ close to the fold point.
Let $ \Delta_a = \{ (x,\rho^2), x \in  I_a \}$ and $\Delta_r= \{ (x,\rho^2), x \in  I_r \}$, where $I_a,I_r$ are intervals on the real line and $\rho$ is sufficiently small.
Futhermore, define $q_a$ to be the point on $\Delta_a$ that belongs to the attracting branch $S^a$, while $q_r$ is equivalently defined as the point on $\Delta_r$ that corresponds to $S^r$. Finally, we are in the position to define the transition map $\pi: \Delta^a \to \Delta^r$, compare to Section \ref{sec: VDP Blowup}.
Following this, \citet{krupa2001} discuss the existence of a critical value for $\lambda$ (denoted $\lambda_c$), where the two branches $S_r$ and $S_a$ must connect in a smooth fashion.  
The transition map $\pi$ has to map the point $p_a$ to $p_r$, if the branches are connected, and the trajectory passing through the fold point is called the maximal canard, see Definition \ref{maxcanard}.
The theorem below describes the technical details involved, and some of the results are derived by the following analysis.
\begin{theorem}[\cite{krupa2001}]
	Assume that system (3.1) satisfies the defining non-degeneracy conditions (Equations \ref{eq: canard sing condition} and \ref{eq: canard non-d condition}) of a canard point. Assume that the maximal solution of the reduced problem connects $ S_a $ to $ S_r $. Then there exists $ \epsilon_0 > 0 $ and a smooth function $\lambda_c(\sqrt{\epsilon})$ defined on $ [0, \epsilon_0] $ such that for $\epsilon \in (0, \epsilon_0)$ the following assertions hold:
	\begin{itemize}
		\item $ \pi(q_{a,\epsilon})=q_{r,\epsilon} $ iff $ \lambda=\lambda_c(\sqrt{\epsilon}) $.
		\item The function $ \lambda_c $ has the expansion
		 \begin{equation*}
			\lambda_c(\sqrt{\epsilon})=-\epsilon\left(\frac{a_1+a_5}{2}+\frac{A}{8}\right)+O\left(\epsilon^\frac{3}{2}\right).
   		 \end{equation*}
			\item The transition map $\pi$ is defined only for $\lambda$ in an interval around $ \lambda_c(\sqrt{\epsilon}) $ of width $ O(\exp(-\frac{c}{\epsilon})) $ fo some $ c>0 $.
			$$ \pd{}{\lambda}(\pi(q_{a,\epsilon})-q_{r,\epsilon})\bigg|_{\lambda=\lambda_c(\sqrt{\epsilon})}>0  $$
	\end{itemize}
\label{Theorem 3.1}
\end{theorem}
%Consider Canard cycles and center manifolds / Freddy Dumortier, Robert Roussarie. for more details on canards in \vdp. ++++++Kieran, what do you mean here+++++
%



\subsection{The Blow-Up Method on the Canard Point}
Now similarly to Section \ref{sec: VDP Blowup}, we consider a transformations of the coordinate system in order to analyse the dynamics in the neighbourhood of the non-hyperbolic equilibrium induced by the canard point. %(++++++ is the eq. induced by lambda??++++).
 The transformations are taken from \citep{krupa2001} and are,
\begin{equation}
x=\bar{r}\bar{x}, \ y=\bar{r}^2y, \ \epsilon=\bar{r}^2\bar{\epsilon}, \ \lambda=\bar{r}\bar{\lambda}.
\end{equation}
Now that we have established these transformation, the charts $K_1$ and $K_2$ can be introduced,  but it is not necessary to consider the third chart, $K_3$. Since the attracting slow manifold connects to the repelling slow manifold, the flow will `bend back' from $K_2$ into $K_1$ instead of leaving the neighbourhood $U$ in the direction of the fast flow, which was described by $K_3$ in Section \ref{sec: VDP Blowup}. 
The trajectory stays close to $S^r$ after passing the fold point. This is very counter-intuitive as the trajectory stays close to a repelling branch.
\begin{figure}[h!]
	\centering
\includegraphics[width=8cm, height=6cm]{Images/vdPhopf-Moment-bendback}
	\caption{The \vdp system for the canard case.}
	\label{fig: flow in canard}
\end{figure}\newpage
Again, equivalently to the procedure in Section \ref{sec: VDP Blowup}, we can define the coordinate transformation for the charts. Note, that in contrast to the generic Blow-Up in Section \ref{sec: VDP Blowup}, the coordinate system is now in $\mathbf{R}^4$, and not in $\mathbf{R}^3$. In chart $K_1$, $y_1=1$, while in $K_2$, $\epsilon_1=1$ and then:
\begin{subequations}
	\begin{align}
	&x=r_1x_1, \ y=r_1^2, \ \epsilon=r_1^2\epsilon_1, \ \lambda=r_1\lambda_1 \label{eq: coordiante K1}\\ 
	&x=r_2x_2, \ y=r_2^2y_2, \ \epsilon=r^2_2, \ \lambda=r_2\lambda_2 \label{eq: coordinate K2}.
	\end{align}
\end{subequations}
Furthermore, we can define the coordinate change between the two charts as follows:
\begin{lemma}
Let $\kappa_{12}$ denote the change of coordinates from $K_1$ to $K_2$. Then $\kappa_{12}$ is given by
\begin{align*}
x_2 = x_1 \epsilon_1^{-1/2}, \ \ \ y_2 = \epsilon_1^{-1}, \ \ \ r_2 = r_1 \epsilon_1^{1/2}, \ \ \ \lambda_2 = \epsilon_1^{-1/2} \lambda_1,
\end{align*}
for $\epsilon_1 >0$.
Similarly $\kappa_{21}=\kappa_{12}^{-1}$ is given by
\begin{align*}
x_1 = x_2 y_2 ^{-1/2}, \ \ \ r_1 = r_2 y_2 ^{1/2}, \ \ \ \epsilon_1 = y_2^{-1}, \ \ \ \lambda_1 = \lambda_2 y_2^{-1/2},
\end{align*}
for $y_2 >0$.
\end{lemma}

We are now in the position to begin with the analysis in the charts, and will first consider chart $K_2$, since, as in Section \ref{sec: VDP Blowup}, $K_2$ holds the most information. 

\subsubsection{Dynamics in \texorpdfstring{$K_2$}{K2}}
Equivalently to Section \ref{sec: VDP Blowup}, we rescale time such that, $\od{r}{t_2}=\od{t}{t_2}\od{r}{t}=\frac{1}{r_2}\od{r_2}{t}$. Then if we substitute the time transformation and Equation  \ref{eq: coordinate K2} into system \ref{eq: canard system} we find, 
%\begin{subequations}
%	\begin{align}
%	r_2^2x_2' - r_2x_2r_2'&=-r_2^2y_2h_1+r^2_2x^2_2h_2,\notag\\
%	\implies x'_2&=-y_2+x_2^2-r_2G_2(x_2,y_2), \label{eq k2 x trans}\\
%	%     \end{aligned}
%	% \end{equation*}
%	% \begin{equation}
%	%     \begin{aligned}
%	r^3_2y_2'-3r_2^2y_2r_2'&=r^2_2(r_2x_2h_4-r_2\lambda_2h_5),\notag\\
%	\implies y_2'&=x_2-\lambda_2+r_2G_2(x_2,y_2), \label{eq: K2 y trans}
%	\end{align}
%	\label{eq: reduced canard k2}
%\end{subequations}
%\noindent where we note that $h_j=h_j(x,y,\epsilon,\lambda)$ for $j=1,2,3,4,5$.  Notice that we have included an additional term in Equation \ref{eq: reduced canard k2} - we define $G_2(x_2,y_2)$ in the following way, $G(x_2,y_2)=(G_1(x_1,y_1),G_2(x_2,y_2))^T=(-\frac{x^3_2}{3},0)^T$. The reason we also define this vector is to aide in the Melnikov computations which we will see later. Then combining this yields our complete system
\begin{subequations}
	\begin{align}
	x'_2&=-y_2+x_2^2-r_2G_1(x_2,y_2) =-y_2+x_2^2-r_2\left(-\frac{x^3_2}{3} \right) ,\label{eq k2 x trans}\\
	y_2'&=x_2-\lambda_2+r_2G_2(x_2,y_2)= x_2-\lambda_2, %\label{eq: K2 y trans}
	\label{eq: K2 y trans}
	\end{align}
\end{subequations}
where $G(x_2,y_2)=(G_1(x_1,y_1),G_2(x_2,y_2))^T=(-\frac{x^3_2}{3},0)^T$. Moreover, \citet{krupa2001} discusses that for this chart we have an interesting result. They note that at $r_2=\lambda_2=0$ the system is integrable which allows us to define a constant of motion $H(x_2,y_2)=\frac{1}{2}\exp{(-2y_2)}\left(y_2-x^2_2+\frac{1}{2}\right)$. For clarity we will first proceed with deriving this equation of motion. Firstly, multiply each equations by, $e^{2y_2}e^{-2y _2}=1$, and define sections of each equation as partial derivatives of $H$ such that, 
\begin{align}
x_2' &=e^{2y_2}e^{-2y_2}( -y_2 +x_2^2 ) =e^{2y_2}\pd{H}{y_2}(x_2,y_2) \\
y_2' &= - e^{2y_2}e^{-2y_2}(-  x_2)=-e^{2y_2}\pd{H}{x_2}(x_2,y_2).
\end{align}
Then we integrate $ \pd{H}{x_2}(x_2,y_2) = -e^{-2y_2} x_2  $ to give,
\begin{align*}
%&\Rightarrow H(x_2,y_2) = \int -e^{-2y_2} x_2 dx\\
 H(x_2,y_2) = - \frac{1}{2} x^2 e^{-2y_2} + C(y),
\end{align*}
where $C(y)$ is the constant of integration, which depends on $y$.
Then, by taking the derivative with respect to $y$ and setting it equal to the expression $\pd{H}{y_2}(x_2,y_2)=e^{-2y_2}( -y_2 +x_2^2 )$, we can find the value for $C(y)$ as follows: 
\begin{align*}
\pd{H}{y_2}(x_2,y_2)&= x^2 e^{-2y_2} + C'(y)\\
%&= e^{-2y_2}( -y_2 +x_2^2 )\\
\Rightarrow C'(y) &= - y_2 e^{-2y_2}
\end{align*}
Finally we integrate $C'(y)$ in order to find an explicit expression for $H$,
\begin{align*}
C(y) = \int - y_2 e^{-2y_2} dy = \frac{1}{2} y_2 e^{-2y_2} + \frac{1}{2} e^{-2y_2} + const,
\end{align*}
using integration by parts.
Then, the final expression is:
\begin{align}
H(x_2,y_2)&=- \frac{1}{2} x^2 e^{-2y_2} + \frac{1}{2} y_2 e^{-2y_2} + \frac{1}{2} e^{-2y_2} + c\\
&= \frac{1}{2}e^{-2y_2}\left(y_2-x^2_2+\frac{1}{2}\right) +c. \label{eq: const of motion}
\end{align}
Note that without loss of generality we can choose $c=0$ because we are interested in the level curves of $H(x_2,y_2)=h$. The reduced system has an equilibrium of center type at the origin. Around this point, the level curves of $H=h$ for $h \in (0,\frac{1}{4})$ define periodic trajectories orbiting the equilibrium. When $h\leq0$, the solutions are unbounded. When $h=0$, we have that $y_2-x_2^2+\frac{1}{2}=0$ and therefore,
\begin{subequations}
	\begin{align}
	x_2'&=\frac{1}{2}\ \	\implies x_2=\frac{t_2}{2}+B, \label{canard: trajectory x}\\
	y_2'&=\frac{t_2}{2}\ \implies y_2=\frac{t_2^2}{4}-\frac{1}{2}, \label{canard: trajectory y}
	\end{align}
\end{subequations} 
where we have directly integrated Equation( \ref{canard: trajectory x}) with respect to the rescaled time $ t_2 $. However, we note that we are able to choose $B=0 $, as we are considering an autonomous (time-invariant) system. Using Equations (\ref{canard: trajectory x}) and (\ref{canard: trajectory y}) we are then able to define $ \gamma_{c,2} $ as, 
\begin{equation}
\gamma_{c,2}(t_2)=(x_{c,2}(t_2),y_{c,2}(t_2))=\left(\frac{t_2}{2},\frac{t^2_2}{4}-\frac{1}{2}\right).   \label{eq: gamma c2}
\end{equation}
This is the candidate trajectory in $K_2$ that perturbs to a maximal canard trajectory. Now that we have established that we must have a flow on the second chart, there must also exist transition maps. Therefore this now enables us to consider the first chart in the following section.


\subsubsection{Dynamics in \texorpdfstring{$K_1$}{K1}}\label{sec:dynamics-in-texorpdfstringk1k1}
For $K_1$ we follow a similar approach to the above. We will use the transformations, 
\begin{equation}
x=r_1x_1, \ y=r_1^2, \ \epsilon=r_1^2\epsilon_1, \ \lambda=r_1\lambda_1 \tag{\ref{eq: coordiante K1}},
\end{equation}
to find the relevant trajectory $\gamma_1$ corresponding to $\gamma_2$ in $K_2$. Now if we first consider the $r_1$ component, 
\begin{align}
2r_1^2r_1'=r_1^2\epsilon_1(r_1x_1-r_1\lambda_1),\notag\\ \label{canard: r1}
\Rightarrow r_1' = \frac{1}{2} \epsilon_1r_1 (x_1 - \lambda_1) = \frac{1}{2} \epsilon_1r_1F,
\end{align}
where $F=F(x_1,y_1,\epsilon_1,\lambda_1)=x_1-\lambda_1$. Next we consider $x=r_1x_1$,
\begin{align*}
r_1r_1'x_1+r_1^2x_1'&=-r_1^2+r_1^2x_1^2,\\
\implies x_1'&=-1+x_1^2-\frac{x_1r_1'}{r_1},
\end{align*}
and substituting in the expression for $r_1'$ results in,
\begin{align}
x_1'=-1+x_1^2-\frac{x_1}{r_1}\left(\frac{r_1\epsilon_1F}{2}\right). \label{eq: canard x1}
\end{align}
We now consider $\epsilon=\epsilon_1r_1^2$ and noting $\epsilon'=0$. Then we have, $r_1^3\epsilon'=-2r_1^2\epsilon_1r_1'$, where we can use Equation (\ref{canard: r1}) to simplify to,
\begin{align}
\epsilon'=-\epsilon_1^2F. \label{canard: epsilon k1}
\end{align}
The last transformation is for the new coordinate $\lambda=r_1\lambda$, noting that $\lambda'=0$. Similarly to the above we find $r_1^2\lambda_1'+r_1\lambda_1r_1'=0$ then, 
\begin{equation}
\lambda'_1=-\frac{\lambda_1\epsilon_1F}{2}, 
\end{equation}
which is a trivial rearrangement as seen in Equation \ref{canard: epsilon k1}. Now if we combine the above we find that the transformed system is of the following form,
\begin{subequations}
	\begin{align}
	r_1'&=\frac{\epsilon}{2}r_1F, \\ \notag
	% \label{canard: r1}
	x_1'&=-1+x_1^2-\frac{x_1\epsilon_1F}{2},\\
	\epsilon'&=-\epsilon_1^2F,\\ \notag
	\lambda'_1&=-\frac{\lambda_1\epsilon_1F}{2}. \notag
	\end{align}
	\label{canard: system of equations}
\end{subequations}
From this system we are now able to make some deductions. We first can observe that the hyperplanes are along the $r_1=\epsilon_1=\lambda_1=0$ which intersect in the invariant line at $l_1=\{(x_1,0,0,0): x_1\in\mathbb{R}\}$ \citep{krupa2001}. As \citet{krupa2001} discusses, the equilibria are located at the end of both branches - Figure \ref{fig: Canard Point} - which are found at $p_a=(-1,0,0,0) \ \text{and} \ p_r=(1,0,0,0)$. We can find the eigenvalues of equation (\ref{canard: system of equations}) for the invariant planes. We find that, 
\begin{equation}
J-\sigma I= \begin{bmatrix}
2x-\sigma & 0 & 0 & 0  \\
0 & -\sigma & 0 & 0&\\
0 & 0 & -\sigma & 0 \\
0 & 0 & 0 & -\sigma
\end{bmatrix},
\end{equation}
which has three zero eigenvalues and one non-zero eigenvalue $\sigma=\pm 2$.This further emphasises that the equilibrium point is non-hyperbolic. 
The dynamics of $K_1$ is very similar to the analysis of $K_1$ in the generic fold case. The center manifold can be applied in order to conclude the persistence of the dynamics for $\epsilon >0$.

\subsection{Full Solution}
The analysis of the charts $K_1$ and $K_2$ provided enough hyperbolicity in order to conclude persistence of the special orbit $\gamma$ for $\epsilon >0$. The orbit $\gamma$ connects the attracting and repelling branch of $S$ at $p_a$ and $p_r$. This can be onserved in Figure \ref{canardsolloc}, which shows the full local dynamics at the fold point.
\begin{figure}[h!]\centering
	\includegraphics[height=8cm,width=10cm]{Images/pres-cancard}
	\caption{Full dynamics at the fold point in the canard case \citep{krupa2001}.}
	\label{canardsolloc}
\end{figure}
We now would like to understand the effect this connection of manifolds a the fold point has on the global dynamics of the system.

\subsection{Effect of the Canard Point}\label{sec:effect-of-the-canard-point}
Now that we have shown that there must exist a flow around the fold point we should now consider the global effect of the canard trajectory. We can see by considering the  system of equations (\ref{eq: extended canard system}) that the equilibrium of the system is at $ (x,y)=(\lambda,\lambda^2[\frac{1-\lambda}{3}]) $  and depends on $\lambda$, as expected. We can find the eigenvalues from the matrix, 
\begin{equation}
A-\sigma I=\begin{bmatrix}
2x-x^2-\sigma&-1&0&0\\
\epsilon&-\sigma&x-\lambda&-\epsilon\\
0&0&-\sigma&0\\
0&0&0&-\sigma
\end{bmatrix}=\sigma^2(\sigma^2+\sigma(x^2-2x)+\epsilon).
\end{equation}
%\begin{equation}
%A-\sigma I=\begin{bmatrix}
%2\lambda-\lambda^2-\sigma&-1&0&0\\
%\epsilon&-\sigma&0&-\epsilon\\
%0&0&-\sigma&0\\
%0&0&0&-\sigma
%\end{bmatrix}.
%\end{equation}
%\textbf{Then our eigenvalues are, $ \sigma=(2-x)x \ \text{and} \ \sigma=0 $, noting that we have an upper triangular matrix. Then we can note that we have a complex eignevalue which causes a Hopf bifurcation, as shown below:}%wrong maths
The eigenvalues of the system are $ \sigma=0 $ and $ \sigma=\frac{2x-x^2\pm\sqrt{(x^2-2x)^2-4\epsilon}}{2} $. Then we consider the values at the equilibrium, $ x=\lambda $, to find that we have a Hopf Bifurcation when $ 4\epsilon>(x^2-2x)^2 $ or when $ \lambda=2 \ \text{or} \ 0 $. The Hopf bifurcation is not a surprising occurrence since it represents a change in stability of an equilibrium. Since the equilibrium of the reduced system changes from stable to relaxation oscillations at the fold points, some bifurcation can be expected at those point.
 This then leads to the following trajectories within the flow - Figure \ref{fig: 4 canard }.

\begin{figure}[h!]
	\centering
	\begin{subfigure}[t]{0.45\textwidth}
		\centering
		\includegraphics[width=.8\linewidth]{vdPhopf-Moment-1.jpg}
		\caption{The initial flow within the system.} \label{fig:timing1}
	\end{subfigure}
	\hfill
	\begin{subfigure}[t]{0.45\textwidth}
		\centering
		\includegraphics[width=.8\linewidth]{vdPhopf-Moment-2.jpg}
		\caption{The flow as it hits the slow manifold.} \label{fig:timing2}
	\end{subfigure}
	
	\vspace{1cm}
	\begin{subfigure}[t]{0.45\textwidth}
		\centering
		\includegraphics[width=.8\linewidth]{vdPhopf-Moment-3.jpg}
		\caption{The flow as it intersects with the fold point.} \label{fig:timing3}
	\end{subfigure}
	\hfill
	\begin{subfigure}[t]{0.45\textwidth}\centering
		% just an empty subfigure to shift C below A
		\includegraphics[width=.8\linewidth]{vdPhopf-Moment-4.jpg}
		\caption{The Hopf bifurcation due to the canard point.}\label{fig:timing4}
	\end{subfigure}\vspace{1cm}
	\begin{subfigure}[t]{0.45\textwidth}\centering
		\includegraphics[width=.9\linewidth, height=6cm]{VDPcanard}
		\caption{Growth of the Hopf bifurcation leading to the formation of a Canard explosion.}
		\label{fig: hopf growth}
	\end{subfigure}
	\caption{The trajectories associated with the canards case of the \vdp system.}
	\label{fig: 4 canard }
\end{figure}\newpage
From Figure \ref{fig: 4 canard } we can see the progression of the flow over the system. From Figure \ref{fig:timing1} we see that the flow starts at an initial condition of $ (x,y)=(1,1) $ and travels along the fast flow towards the attracting branch. Then in Figure \ref{fig:timing2} the flow has hit the attracting branch, where it then follows along the slow flow towards the fold point at $ (x,y)=(0,0) $, which is described by Figure \ref{fig:timing3}. Then from Figures \ref{fig:timing3} and \ref{fig:timing4} we can observe the Hopf bifurcation. These small amplitude oscillations are the pertubation of the closed orbits due to the constant of motion found in the singular limit in chart 2.  It can be seen that the trajectories follow the repelling branch (see Figure \ref{fig: Canard Point}) until the flow is sufficiently far from the fold point where it will then repel towards the attracting branch - Figure \ref{fig:timing4}. For increasing values of $\lambda$, the canard sycles get larger and trajectories that follow the repelling branch are either repelled towards the left or the right - Figure \ref{fig: hopf growth}. The canards jumping off the repelling branch to the right are called canards with head because of the shape they trace out.
The question whether trajectories jump off to the left or to the right is answered by investigating the separation of the manifolds.


\subsubsection{Separation of the Manifolds}\label{sec:separation-of-the-manifolds}
\begin{figure}[h!]\centering
	\includegraphics[height=8cm,width=10cm]{Images/Separation}
	\caption{Separation of $ M_a $ and $ M_r $ \citep{Kuehn}.}
	\label{fig: splitting}
\end{figure}%\newpage
%\textbf{Discuss splitting on the manifold}
The maximal canard connects the attacting and repelling branches of the slow manifold as shown in Figure \ref{fig: splitting}, where the trajectory is denoted by $\gamma_c$. However, for other values of $\lambda$ than the critical value, the attracting and repelling branches are no longer connected, as show in Figure \ref{fig: splitting}. To measure the distance between the manifolds we will apply a Melnikov Computation - see  \citet{krupa2001} for direct use. To discover whether we have a splitting between the attracting and repelling branches we need to consider the $ y $ coordinates in the second chart \st $ y_{a,2}(0)-y_{r,2}(0) $ is a distance function, which can be written as $ D_c(r_2,\lambda_2)=H(0,y_{a,2}(0))-H(0,y_{r,2}(0)) $. Note that $ \pd{}{y_2}H(0,y_2)\neq 0 $ \citep{krupa2001}. From here we can use the following proposition,
\begin{prop}
	[\citealp{krupa2001}]
	For a small enough $ \rho $ and $ \mu $ the distance function has the expansion
	\begin{equation*}
	D_c(r_2,\lambda_2)=d_{r_2}r_2+d_{\lambda_2}\lambda_2+O(2),
	\end{equation*}
	with 
	\begin{subequations}
		\begin{align}
		d_{r_2}&=\int_{-\infty}^{\infty}(\nabla H(\gamma_{c,2}(t)))^T\cdot G(\gamma_{c,2}(t))dt,\\
		d_{\lambda_2}&=\int_{-\infty}^{\infty}(\nabla H(\gamma_{c,2}(t)))^T\cdot (0,-1)^T,
		\end{align}
	\end{subequations}
	where the matrix $ G(\gamma_{c,2}(t)) $ is as defined in Section \ref{sec:dynamics-in-texorpdfstringk1k1} and $ \gamma_{c,2} $ is the trajectory associated with the singular canard. 
\end{prop}
Then, following the proof provided by \citet{krupa2001}, we find that we will have a split occurring between our branches if the canard falls outside of the domain of order $ O(e^{-\frac{c}{\epsilon}}) $ \st $ D_c(r_2,\lambda_2)\neq 0 $. We can calculate these explicitly for the the \vdp system by using Equations \ref{eq: const of motion}, \ref{eq: gamma c2} and $ G(x_2,y_2)=(-\frac{x_2^3}{3},0)^T $. Consequently we have,

\begin{align*}
d_{r_2}&=\int_{-\infty}^{\infty} \nabla\left(\frac{1}{2}e^{-2y_2}\left(y_2-x^2_2+\frac{1}{2}\right)\right)^T\cdot\left((-\frac{x_2^3}{3},0)^T\right)\bigg|_{\gamma_{c,2}} dt,\notag\\
&=-\frac{e}{16}\int_{-\infty}^{\infty}t^4e^{-\frac{T^2}{2}}dt,
\end{align*}
noting $ \nabla(H(\gamma_{c,2}(t)))= \exp(1-\frac{t^2}{2})\left[-1,\frac{1}{2}\right]$. Now the task is to show that $ d_{r_2} $ is finite, we do this by using integration by parts to find,
\begin{equation}
d_{r_2}=	\frac{e}{16}\int_{-\infty}^{\infty}e^{-\frac{T^2}{2}}dt=\frac{e\sqrt{2\pi}}{16}<\infty,
\end{equation}
where we note that we have made use of the Gaussian Integral \citep{GausIntegral} and $d_{r_2}$ is finite. We then apply an analogous approach to $d_{\lambda_2} $ \st, 
\begin{align}
d_{\lambda_2}&=\int_{-\infty}^{\infty} \nabla\left(\frac{1}{2}e^{-2y_2}\left(y_2-x^2_2+\frac{1}{2}\right)\right)^T\cdot\left((0,-1)^T\right)\bigg|_{\gamma_{c,2}} dt, \notag\\
&=-\frac{e}{2}\int_{-\infty}^{\infty} e^{-t^2}dt=-\frac{e\sqrt{2\pi}}{2}<0,
\end{align}
where we have used the same techniques as above and we can conclude that $ d_{\lambda_2} $ is also finite. Combining the above yields that the distance function is,
\begin{equation}
D_c(r_2,\lambda_2)=\frac{e\sqrt{2\pi}}{16}r_2-\frac{e\sqrt{2\pi}}{2}\lambda_2+O(2),
\end{equation}  
whereby we have that the manifolds in the \vdp system split for $ \lambda_2\neq\frac{r_2}{8} $. Otherwise the attracting and repelling branches are connected and the maximal canard exists. If the manifold splits then we would find that the manifold is similar to Figure \ref{fig: splitting} whereby the flow will either jump off to the right - see Figure \ref{fig: hopf growth} - or the flow will be trapped in the canard region and then be repelled back to the attracting manifold, as we see with our connected system - Figure \ref{fig: 4 canard }. 
The presence of either of these cases depends on how the manifold breaks.
This describes the global dynamics for the region close to $\lambda =0$, $\lambda=2$.
Therefore, we are now in the position to have a description of the full dynamics of the two dimensional Van der Pol System for all parameter regimes of $\lambda$.
Since the canard solutions are restricted to an exponentially small region around the fold points, we now move on to three dimensional fast-slow systems, in which canards are more common occurrences.




























