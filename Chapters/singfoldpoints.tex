 (ref : book kuehn)
One of the requirements of Fenichel's Theorem is normal hyperbolicity. However, fast-slow systems can display singular points where normal hyperbolicity is no longer given and therefore the conclusions of (\ref{Fenichel}) no longer hold at these singularities.
Singularities in the setting of fast-slow systems are points $(x_0,y_0)$ on the critical manifold $S_0$, for which the Jacobian $ \frac{\partial f}{\partial x}(x_0,y_0, \lambda, 0)$, has one or more eigenvalues with zero real part. Comparing this with Definition \ref{NormHyp} shows that this is a negation of normal hyperbolicity. Singularities are points where trajectories can jump between fast and slow flow. 

The simplest of those singularities is called a fold point, which is defined as follows:
\begin{definition}{\textbf{Fold Point}} \label{FoldDef} \\
A fold point $(x_0,y_0) \in S_0$ is a point where the Jacobian $ \frac{\partial f}{\partial x}(x_0,y_0, \lambda, 0)$ has only one eigenvalue with zero real part.
\end{definition}
At the fold point, the system (\ref{FastS0}) undergoes a saddle-node bifurcation. (+++explain?++++)
The fold point is non-degenerate if it satisfies the non-degeneracy assumptions:
\begin{align} \label{NonDeg}
\begin{cases}
\frac{ \partial ^2 f}{ \partial x^2} (x_0,y_0, \lambda, 0) \neq 0 \\
\frac{\partial f}{\partial y}(x_0,y_0, \lambda, 0) \neq 0.
\end{cases}
\end{align}
Furthermore, if $(x_0,y_0)$ satisfies the transversality condition $g(x_0,y_0, \lambda, 0) \neq 0$, then it is called a generic fold point.
For these generic folds there exists a theorem that states that the slow flow on $S_\epsilon$ (?) near $(x_0,y_0)$ has either positive or negative sign, implying that no equilibria of the slow flow are close to $(x_0,y_0)$. Therefore, for generic fold points no canards will be observed, which is a relevant observation for Section \ref{Canards}.
The analysis of fold points is using a method called Blow Up Method, which is discussed in Section \ref{sec: VDP Blowup}.

In systems containing generic fold point a certain behaviour of the flow can be observed, called Relaxation Oscillations. These are defined as follows:
\begin{definition}{\textbf{Relaxation Oscillation}}\\
A periodic trajectory $\gamma_\epsilon$ is the relaxation oscillation of the fast-slow system if the following holds:
In the singular limit there exists a trajectory $\gamma_0$, which alternates between fast and slow bits and describes a closed loop in the system. This trajectory $\gamma_0$ persists as $\gamma_\epsilon$ under a small pertubation $\epsilon >0$.
\end{definition} 
Systems containing non-generic folds or other types of singularities can display different types of periodic orbits.
