This project will be considering the Fast-Slow dynamics in non-linear ordinary differential equations. The project will start by considering the theory associated with Fast-Slow dynamical systems such as Geometric Perturbation Theory (Section \ref{GSPT}). Then the project moves onto two-dimensional systems by first looking at the general form, using \citet{krupa2001} theory (Section \ref{Intro}), before applying it to the \vdp system (Section \ref{sec:the-van-der-pol-equation}). From here the project will also consider the non-hyperbolicity of the fold points present in the system, where a jump might occur (Section \ref{sec: VDP Blowup}). Once the system has been `blown-up' for the normal case, it is prudent to consider the canard system - where the parameter $ \lambda $ is introduced. This could cause a split within the manifolds or a Hopf bifuraction could occur causing a periodic solution (Sections \ref{sec:canard-points} and \ref{sec:separation-of-the-manifolds}). Once this has been considered, the next step is to consider the Fast-Slow system in a three dimensional case for folded singularities (Section \ref{sec: threedimfolds}) before continuing onto the theory behind Mixed Mode Oscillations (Section \ref{sec: MMO Oscilaltions}). \textbf{Lastly, the project will discuss the numerical simulations associated with the construction of the models and the results obtained (Section)}.