%This project will be considering the Fast-Slow dynamics in non-linear ordinary differential equations. The project will start by considering the theory associated with Fast-Slow dynamical systems such as Geometric Perturbation Theory (Section \ref{GSPT}). Then the project moves onto two-dimensional systems by first looking at the general form, using \citet{krupa2001} theory (Section \ref{Intro}), before applying it to the \vdp system (Section ??). From here the project will also consider the non-hyperbolicity of the fold points present in the system, where a jump might occur (Section \ref{sec: VDP Blowup}). Once the fold point has been `blown-up' for the normal case, it is prudent to consider the canard system - where the parameter $ \lambda $ is introduced. This could cause a split within the manifolds or a Hopf bifuraction could occur causing a periodic solution (Sections \ref{sec:canard-points} and \ref{sec:separation-of-the-manifolds}). Once this has been considered, the next step is to consider the Fast-Slow system in a three dimensional case for folded singularities (Section \ref{sec: threedimfolds}) before continuing onto the theory behind Mixed Mode Oscillations (Section \ref{sec: MMO Oscilaltions}). \textbf{Lastly, the project will discuss the numerical simulations associated with the construction of the models and the results obtained}.
%
%
%
%This project will be considering the Fast-Slow dynamics in non-linear ordinary differential equations. The aim of the project is to develop the techniques required to analyse a dynamical system in full detail, due to the presence of problematic points. These points are often refereed to as fold points within the system. By this it is meant that within a system there exists fold points (singularities) which causes standard Geometric Perturbation Theory to fail motivating the analysis within the project. This failure forces us to consider alternative methods when considering the full system. One such method that we discuss in length is the idea of Blow-up by which we are, in essence, magnifying our fold point to find the flow in this region. The results of this can be seen throughout the report as we would expect periodic solutions or jumps within the system. Following the development of the theory we then apply these methods to the \vdp system, whereby we consider the case for $ \lambda=1 $ and $ \lambda>0 $ - leading to the formation of a canard solution in the latter case. As a result this will give the reader a good base in Fast-Slow systems which can then be applied to three dimensional cases, such as Mixed Mode Oscillators (MMOs). For this we consider the case of a folded node and discuss the existence of Hopf bifurcations in the singular case.


This project will be considering the Fast-Slow dynamics in non-linear ordinary differential equations. The aim is to introduce techniques required to analyse a system in full detail. The main obstruction will be the existence of singular points within the system, so called fold points. These cause standard Singular Geometric Perturbation Theory to fail, so alternative methods are required to explain the global dynamics. We discuss the blow-up method by which we are, in essence, magnifying the fold point to find the flow in this region. The results of this can be seen throughout the report as we would expect periodic solutions or jumps within the system. The methods are then applied to the \vdp system, whereby we consider the case for $ \lambda=1 $ and $ \lambda>0 $ - leading to the formation of a canard solution in the latter case. Three dimensional systems and Mixed Mode Oscillators (MMOs) are also introduced. For this we consider the case of a folded node and discuss the existence of Hopf bifurcations in the singular case.  