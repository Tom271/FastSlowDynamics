
%\section{Folded Singularities in three dimensions}
Now that we have considered the two dimensional case for a folded singularity we can extend it to a third dimension in our system. This can be done by considering a system of one fast and two slow variables \st, 
\begin{equation}
\begin{cases}
\epsilon \dot{x}&=f(x,y,z,y,\epsilon),\\
\dot{y}&=g_1(x,y,z,y,\epsilon),\\
\dot{z}&=g_2(x,y,z,y,\epsilon),
\end{cases}
\end{equation}
which we can see is an extension of our original form - Equation \ref{SlowS} \citep{MMO}. Furthermore, \citet{MMO} also discusses that the addition of an extra slow variable causes issues with respect to the existence of a canard solution. This is because our existence ranges increases from $ O(\epsilon) $ to $ O(1) $, noting that $ \epsilon\ll 1 $ \citep{MMO}. Then for this system we are able to make similar assumptions to the previous case, Section \ref{sec:singularities-and-fold-points}, but it is obvious we now must have more than one fold point. We can see that this is the case in Figure \ref{fig: 3d folded singularity},
\begin{figure}[h!]\centering
	\includegraphics{}
	\caption{Three dimensional folded singularity.}
	\label{fig: 3d folded singularity}
\end{figure}
as our fold point now can take multiple locations within our system. From here we are able to define some non-degeneracy conditions, much like we did in Section \ref{intro},
\begin{equation}
\begin{aligned}
&f(p_*,\lambda,0)=0,\\
&\pd{}{x}f(p_*,\lambda,0)=0,\\
&\pd{^2}{x^2}f(p_*,\lambda,0)\neq 0,\\
&D_{(y,z)}f(p_*,\lambda,0) \ \text{has full rank one},
\end{aligned}
\label{eq: non-degeneracy 3d system}	
\end{equation}
where we denote $ p_*=(x_*,y_*,z_*)\in F $ as our fold points and $ D_{(y,z)} $ as our Jacobian \wrt $ y \ \text{and}\ z $ \citep{MMO}. In addition to this we can see from Figure \ref{fig: 3d folded singularity} that we have some interesting flows within our system. These flows do not follow the standard pattern as we saw in Figure \ref{fig: vdp flow diagram}, instead the slow flow switches its orientation when the flow hits the fold point and continue to flow in that direction, as a desingularised flow - these are called isolated singularities \citep{MMO}. %dot in diagram
As a result we are able to express these flows in the following manner, using Equation \ref{eq: non-degeneracy 3d system}, 
\begin{equation}
\begin{cases}
\dot{x}&=g_1\pd{f}{y}+g_2\pd{f}{z}\\
\dot{y}&=-g_1\pd{f}{x},\\
\dot{z}&=-g_2\pd{f}{x},
\end{cases}
\end{equation}
where we can then define a folded singularity if $ g_1(p_*,\lambda,0)\pd{}{y}f(p_*,\lambda,0)+g_2(p_*,\lambda,0)\pd{}{z}f(p_*,\lambda,0)=0 $, for our flow on branches ($ S $) \citep{MMO}. Next we need to consider the stability of our fold points. We do this by constructing the Jacobian of our system, 
\begin{equation}
J=\begin{bmatrix}
\pd{\dot{x}}{x}&\pd{\dot{x}}{y}&\pd{\dot{x}}{z}&\pd{\dot{x}}{\lambda}&\pd{\dot{x}}{\epsilon}\\
\pd{\dot{y}}{x}&\pd{\dot{y}}{y}&\pd{\dot{y}}{z}&\pd{\dot{y}}{\lambda}&\pd{\dot{y}}{\epsilon}\\
\pd{\dot{z}}{x}&\pd{\dot{z}}{y}&\pd{\dot{z}}{z}&\pd{\dot{z}}{\lambda}&\pd{\dot{z}}{\epsilon}\\
\end{bmatrix}
\end{equation}
