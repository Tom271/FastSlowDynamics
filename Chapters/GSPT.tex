In this section the singular limit is analysed by considering the reduced dynamics of the fast-slow system when $\lambda=1$ and identifying the points where the dynamics of the singular system, $\epsilon \to 0$, is not known. Moving on from that problem, the standard theory, called Geometric Singular Perturbation Theory, for concluding the persistence of the singular dynamics for the full system is introduced. Following this, the points where this theory does not hold are established.

\subsection{Reduced Dynamics}
In order to determine the reduced dynamics on the critical manifold $S$, we consider the reduced problem (\ref{SlowS0}). 

Rearranging the first equation, $f=0$, of System \ref{SlowS0}, define $\phi$ as the cubic $ \phi(x) = x^2-\dfrac{x^3}{3}$ can be defined. Then using the chain rule, we can express the change in $x$ using the second equation of (\ref{SlowS0}) as,
\begin{equation}
\phi_x(x)\dot{x}=g(x,\phi(x),0).
\label{eq: general reduced}
\end{equation}
Rearranging this gives an expression for the dynamics in $x$ on $S$.
The derivative \wrt $x$ gives $\phi_x(x)=2x-x^2$, therefore Equation \ref{eq: general reduced} becomes, 
\begin{align*}
\dot{x} = \frac{g(x,\phi(x),0)}{ \phi_x(x)} = \frac{ x-1}{2x-x^2} =\frac{ x-1}{x(2-x)}.
\end{align*}
The reduced dynamics are singular at $x=0$ and $x=2$. Therefore, no conclusions about the dynamics of $x$ can be made at the jump points in this system. \\ 

The dynamics in the layer problem (\ref{FastS0}) are simpler. In this system, the manifold $S$ contains all the equilibria. Whatever the initial data, the position moves horizontally towards $S$ at which point it stops. The issue of reconciling the reduced and layer problems is considered using geometric singular perturbation theory (GSPT) below, along with conditions for the persistence of the dynamics under perturbation in $\epsilon$. However, this will not cover the singularities. The reasoning for this is given later on in this section. The blow up method is employed in Section \ref{sec: VDP Blowup} to address the singularities, as well as the problems encountered when trying to apply GSPT. 

\subsection{Geometric Singular Perturbation Theory}	
The main question GSPT aims to answer is the following: under what conditions can it be concluded that the dynamics on the critical manifold $S=S_0$, persist as an invariant manifold $S_{\epsilon}$ under a small perturbation $0<\epsilon\ll 1$? The main contribution to GSPT comes from \citet{Fenichel}  and his three theorems can be summed up in one, according to \citet{MMO}. However, before stating the theorem, some formal definitions are needed.

\begin{definition}{\textbf{Normal Hyperbolicity \citep{firstpaper}}} \label{NormHyp}
	\\
	A submanifold $M \subseteq S$ is called normally hyperbolic, if the Jacobian $ \frac{\partial f}{\partial x}(x,y, \lambda, 0),$ where $(x,y) \in M$, has only eigenvalues with nonzero real part.
\end{definition} 

Moreover, the points $(x,y) \in M$, $M$ normally hyperbolic, are hyperbolic equilibria of Equation \ref{FastS0} \citep{MMO}.
A normally hyperbolic submanifold can be classified according to its stability property: If $M$ only has eigenvalues with positive real part it is called repelling, otherwise eigenvalues with negative real part are called attracting and if $M$ is neither attracting nor repelling it is called a saddle-type submanifold \citep{MMO}. Furthermore, stable and unstable manifolds can be defined as $W^s(M)$  and $W^u(M)$, corresponding to the eigenvalues with negative and positive real part, respectively. Now, with the following definition it is established which notion of distance is going to be employed throughout this analysis.

\begin{definition}{\textbf{Hausdorff Distance \citep{Kuehn}}}\\
	The Hausdorff Distance of two nonempty sets $V,W \subset \mathbf{R}^n$, for some $n \in \mathbf{N}$ 
	is defined as 
	\begin{align*}
	d_H(V,W)= \max \{ \sup_{v \in V} \inf_{w \in W} || v- w ||, \sup_ {w \in W}\inf_{v \in V} || v- w ||\}.
	\end{align*}
%	(ref: book kuehn)
\end{definition}

Now combining the above we can state Fenichel's Theorem.

\begin{theorem}[\textbf{Fenichel's Theorem} \citep{MMO}] \label{Fenichel}	
	Suppose $M=M_0$ is a compact, normally hyperbolic submanifold  (possibly with boundary) of the critical manifold $S$ Equation \ref{CriticalS} 	and  that $f, g \in C^r, r < \infty $. Then for $\epsilon >0$, sufficiently small, the following holds:\\
	(F1) There exists a locally invariant manifold $M_{\epsilon}$, diffeomorphic to  $M_0$. Local invariance means that $M_{\epsilon}$ can have boundaries through which trajectories enter or leave.\\
	(F2) $M_{\epsilon}$ has a Hausdorff distance of $O(\epsilon)$ from $M_0$.\\
	(F3) The flow on $M_{\epsilon}$  converges to the slow flow as $\epsilon \to 0$.\\
	(F4) $M_{\epsilon}$ is $C^r$- smooth.\\
	(F5) $M_{\epsilon}$ is normally hyperbolic and has the same stability properties with respect to the fast variabes as $M_0$ (attracting, repelling or saddle type).\\
	(F6) $M_{\epsilon}$ is usually not unique. In regions that remain at a fixed distance from the boundary of  $M_{\epsilon}$, all manifolds satisfying (F1)-(F5) lie at a Hausdorff distance $O(e^{-K/\epsilon})$ from each other for some $K>0$ with $K=O(1)$.\\
	The normally hyperbolic manifold $M_0$ has associated local stable and unstable manifolds
	\begin{align*}
	W^s(M_0) =\cup_{p \in M_0} W^s(p) \textrm{\ \ and\ \ } W^u(M_0) =\cup_{p \in M_0} W^u(p),
	\end{align*}
	where  $W^s(p)$ and $W^u(p)$ are the local stable and unstable manifolds of $p$ as a hyperbolic equilibrium of the layer equations, respectively. These manifolds also persist for $\epsilon > 0$, sufficiently small: there exist locally stable and unstable manifolds $W^s(M_\epsilon)$ and $W^u(M_\epsilon)$, respectviely, for which conclusions (F1) - (F6) hold if we replace $M_\epsilon$ and $M_0$ by  $W^s(M_\epsilon)$ and $W^s(M_0)$ (or similarly by  $W^u(M_\epsilon)$ and $W^u(M_0)$).
\end{theorem} 
Fenichel's Theorem establishes that the submanifold, $M_0$, of the critical manifold, $S_0$, persists as slow manifold $M_\epsilon$ as $\epsilon >0$, given it is compact and normally hyperbolic. The theorem furthermore establishes that the stable and unstable manifolds persist as well as the individual fibres, namely $W^s(p)$ and $W^u(p)$, that are associated to each base point $p \in M_0$.
Therefore, under the assumptions of the theorem, the flow of the Fast-Slow system remains $O(\epsilon)$ close to the flow of the system in the singular limit $\epsilon \to 0$.\\

The importance of this result lies in the fact that the behaviour of the full system can be analysed by looking at the system in the singular limit instead. The main assumption that has to be satisfied in order to apply Theorem \ref{Fenichel} is normal hyperbolicity. The points where this assumption fails is discussed in the following section.


\subsection{Fold Points}\label{sec:singularitiesandfoldpoints}
One of the requirements of Fenichel's Theorem is normal hyperbolicity \citep{Kuehn}. However, Fast-Slow systems can display singular points where normal hyperbolicity is no longer given and therefore the conclusions of Theorem \ref{Fenichel} no longer. These singularities are points where trajectories can jump between fast and slow flow. The singularities in the setting of Fast-Slow systems are points $(x_0,y_0)$ on the critical manifold $S_0$, for which the Jacobian has one or more eigenvalue with zero real part. Comparing this with Definition \ref{NormHyp} shows that this is a negation of normal hyperbolicity. The simplest of those singularities are called fold points, which is defined as follows:
\begin{definition}{\textbf{Fold Point}} \label{FoldDef} \\
	A fold point $(x_0,y_0) \in S_0$ is a point where the Jacobian $ \frac{\partial f}{\partial x}(x_0,y_0, \lambda, 0)$ has only one eigenvalue with zero real part.
\end{definition}
%At the fold point, the system (Equation \ref{FastS0}) undergoes a saddle-node bifurcation. (+++explain?++++)
Moreover we say that the fold point is non-degenerate if it satisfies the non-degeneracy assumptions,
\begin{align} \label{NonDeg}
\begin{cases}
\frac{ \partial ^2 f}{ \partial x^2} (x_0,y_0, \lambda, 0) \neq 0, \\
\frac{\partial f}{\partial y}(x_0,y_0, \lambda, 0) \neq 0.
\end{cases}
\end{align}
Furthermore, if $(x_0,y_0)$ satisfies the transversality condition $g(x_0,y_0, \lambda, 0) \neq 0$, then it is called a generic fold point.
For these generic folds there exists a theorem that states that the slow flow on $S_\epsilon$ near $(x_0,y_0)$ has either positive or negative sign, implying that no equilibria of the slow flow are close to $(x_0,y_0)$. Therefore, for generic fold points no canards will be observed, which is a relevant observation for Section \ref{sec:canard-points}. First, we must find the fold points in the system.\\
\subsection{The Van der Pol System: GSPT and Fold Points}\label{VDPfolds}
The aim is to establish normal hyperbolicity of the reduced system in order to conclude that the manifold $S_0$ perturbs to a manifold $S_\epsilon$. Since Theorem \ref{Fenichel} concludes the persistence of the stable and unstable manifolds of $S$ as well, we do not need to consider the normal hyperbolicity of the layer problem separately. 

In order to analyse the system (\ref{SlowS0}), we note from the previous analysis that the expression for $\dot{x}$ is dependent on the expression for $\dot{y}$. Therefore, it suffices to consider a one dimensional Jacobian,
\begin{align*}
J= \frac{\partial f}{\partial x} = 2x - x^2.
\end{align*}
Then the eigenvalues corresponding to this are $\sigma = x(2-x)$. These are zero at $x=0$ and $x=2$. At these points normal hyperbolicity is lost and Theorem \ref{Fenichel} cannot be applied.
Therefore, the critical manifold has to be divided as follows:
\begin{align*}
S^a &=\bigg \lbrace (x,y): y = x^2-\frac{x^3}{3}, x< 0 \bigg \rbrace \cup \bigg \lbrace (x,y): y = x^2-\frac{x^3}{3}, x>2 \bigg\rbrace \\
S^r &= \bigg\lbrace (x,y): y = x^2-\frac{x^3}{3}, 0< x< 2 \bigg\rbrace,
\end{align*}
such that $S^a \cup S^r \cup \{0\} \cup \{2\} = S$.
The manifolds $S^a_0$ and $S^r_0$ are normally hyperbolic everywhere and  Theorem \ref{Fenichel} (Fenichel's) can be applied in order to conclude the persistence of the manifold as slow manifolds $S^a_\epsilon$ and $S^r_\epsilon$. At the points $x=0$ and $x=2$ the normal hyperbolicity is not given, since the eigenvalue associated to $S$ is zero at these points.

The next step of the analysis is to classify the nonhyperbolic points and conclude that they are generic folds.
The points of interest are $(x_0^1,y_0^1)=(0,0)$ and $(x_0^2,y_0^2)=\left(2,\dfrac{4}{3}\right)$. By Definition \ref{FoldDef}, there is only one eigenvalue with zero real part at $(x_0,y_0)$.
These points are nondegenerate if the non-degeneracy assumptions (Equation \ref{NonDeg}) hold:
\begin{align*}
\begin{cases}
\frac{ \partial ^2 f}{ \partial x^2} (x_0^1,y_0^1, 0) = 2-2 x_0^+ = 2 \neq 0 \\
\frac{\partial f}{\partial y}(x_0^1,y_0^1, 0) = -1 \neq 0,
\end{cases}
\end{align*}
and equivalently for the other fold point
\begin{align*}
\begin{cases}
\frac{ \partial ^2 f}{ \partial x^2} (x_0^2,y_0^2,0) = -2 x_0^2 =4\neq 0 \\
\frac{\partial f}{\partial y}(x_0^2,y_0^2, 0) = -1 \neq 0.
\end{cases}
\end{align*}
Therefore, the two fold points are non-degenerate.
Futhermore, it can be checked if a fold point is generic. It then has to satisfy the transversality condition $g(x_0,y_0,0) \neq 0$.
The two fold points considered here are generic, since 
\begin{align*}
g(x_0^1,y_0^1,0)= -1\neq 0 \\
g(x_0^2,y_0^2,0)= 1 \neq 0.
\end{align*}
Note that systems containing non-generic folds or other types of singularities can display different types of periodic orbits.
\\
\\
In the analysis of the reduced system it became apparent that the fold points are singularities of the reduced flow on $S_0$, and therefore the dynamics in the singular limit cannot be determined. Furthermore, Fenichel Theory does not apply at the folds because normal hyperbolicity breaks down at these points, as discussed above. Therefore, even if the dynamics around the folds in the singular limit were known, no conclusions could be drawn for the perturbed system with $S_\epsilon$.
Alternative methods have to be employed to describe the dynamics on the fold points in the singular limit and furthermore to be able to conclude the dynamics of the full system at the fold points from this analysis.
The method considered for analysis is called the Blow-Up Method.




