
One fast-slow system that contains generic fold points and therefore displays relaxation oscillations is called the Van der Pol System. This can be derived from the Van der Pol Oscillator, which is a well-studied second order ODE that is used to model a variety of physical and biological phenomena. It was developed by the dutch physicist and electrical engineer Balthasar Van der Pol, who conducted research on electrical circuits, in which he observed stable oscillations, later named relaxation oscillations.
The derivation of the Van der Pol fast-slow system of the form (\ref{FastS}) is presented in the following section.

\subsection{Derivation of the Van der Pol Fast-Slow System}

The Van der Pol Oscillator describes the evolution of the position coordinate \(x(t)\) according to the following the ODE:
\begin{equation} \label{eq:vdP}
\ddot{x}(t)-\mu\left(1-x^2(t)\right)\dot{x}(t)+x(t)=0,   
\end{equation}
where \(\mu \gg 1\) is a scalar constant. \par  

A new variable\(w=\dot{x}+\mu F(x)\) is introduced, where \(F(x)=\frac{x^3}{3}-x\). $F$ is chosen such that  \(F'(x)=-(1-x^2)\) is the nonlinear term in Equation \ref{eq:vdP}. Differentiating \(w\) we obtain
\begin{align*}
    \dot{w}&=\ddot{x}+\mu\od{}{x}\left(\frac{x^3}{3}-x\right)\od{x}{t}\\
    & =\ddot{x}+\mu(x^2-1)\dot{x}\\
    &= -x
\end{align*}
Here, the last equality follows from rearranging (\ref{eq:vdP}). We now have a two dimensional system:
\[\begin{cases} \dot{x}=w-\mu F(x)\\
 \dot{w}=-x\end{cases}\]
and letting \(y=\frac{w}{\mu}\) results in
 \[\begin{cases} \dot{x}=\mu\left(y-F(x)\right)\\
 \dot{y}=-\frac{x}{\mu}.\end{cases}\]

Now, using a rescaling of time $ \tilde{t} = \mu \tau$ and setting $ \frac{1}{\mu^2} = \epsilon$ results in the system: \\
+++$ \tilde{t}$ is the original variable, we transform into the slow system but state the fast system first because thats the order we always have them in. slightly confusing. ideas? Also. Need to define $\lambda$ as either zero or 1 depending on where to mention it...++++++++
\begin{equation}\label{fastsystem}
    \begin{cases} x'=y-\frac{x^3}{3}+x\\
    y'=-\epsilon x,
    \end{cases}
\end{equation}
which is of the form (\ref{FastS}), the fast system, and the rescaling of time $t= \epsilon \tau$ results in 
\begin{equation}\label{slowsystem}
    \begin{cases} \epsilon \dot{x}=y-\frac{x^3}{3}+x\\
    \dot{y}=-x,
    \end{cases}
\end{equation}
which is in the form of (\ref{SlowS}), the slow system.

As in Sechtion \ref{Intro} the fast and slow system can be analysed by considering the limiting case $\epsilon \to 0$.
The two systems then become
\begin{equation}\label{fastsystem0}
    \begin{cases} x'=y-\frac{x^3}{3}+x\\
    y'=0,
    \end{cases}
\end{equation}
which is of the form (\ref{FastS0}), the layer problem, and the reduced problem 
\begin{equation}\label{slowsystem0}
    \begin{cases} 0=y-\frac{x^3}{3}+x:=f\\
    \dot{y}=-x.
    \end{cases}
\end{equation}

\subsection{Phase Plane Analysis (is it?? some part 'singularity analysis')}
\begin{figure}[h]\centering
%	\includegraphics{}
%	\caption{Flows in our \vdp system.}
%	\label{\ref{fig: vdp flow diagram}}
\end{figure}
Considering (\ref{fastsystem0}), it can be observed that the flow is dominated by the dynamics in $x$ which is cubically depending on $x$.
Furthermore, it is clear that in the layer problem the dynamics in $y$ are constant and therefore the flow is horizontal and is only influenced by $y$ as a constant parameter. Then $x$ is called the fast variable.
This is immediately obvious when comparing this to the reduced problem (\ref{slowsystem0}), where the flow is restricted to $f=0$, which is in the form of a cubic function. This defines a critical manifold. Restricted to this manifold, the flow is dominated by the dynamics in $y$, which linearly depends on $x$, which is much slower than the cubic dependence in the layer problem. Therefore, this is called the slow flow and $y$ is the slow variable. \\


The aim of this analysis is to be able to analyse the system in the singular limit $\epsilon \to 0$ and apply appropriate theory to conclude the persistence of the dynamic for $\epsilon > 0$. 
Section \ref{GSPT} introduced one instance where this persistence can be concluded.
The main requirement for the theory in Section \ref{GSPT} is normal hyperbolicity of the critical manifold.
Considering the manifold $C_0= \{ (x,y) : 0=y-\frac{x^3}{3}+x:=f \}$,the Jakobian $\frac{ \partial f}{\partial x}(x,y,0) = -x^2 + 1$, which has a zero real part at $x_0= \pm 1$. Together with the corresponding $y_0$ are singularities of the system.
Further analysis has to be done below in order to conclude that they are generic fold points.
The points of interest are $(x_0^+,y_0^+)=(1,-\dfrac{2}{3})$ and $(x_0^-,y_0^-)=(-1,\dfrac{2}{3})$.

By Definition \ref{FoldDef}, there is only one eigenvalue with zero real part at $(x_0,y_0)$. Evaluating the Jakobian at each of the points in turn shows:
\begin{align*}
\begin{cases}
 \frac{ \partial f}{\partial x}(x_0^+,y_0^+,0) = -1^2 + 1 =0 \\
 \frac{ \partial f}{\partial x}(x_0^-,y_0^-,0) = -(-1)^2 + 1 =0,
\end{cases}
\end{align*}
where each of the zeros are simple.
Therefore $(x_0^+,y_0^+)$ and $(x_0^-,y_0^-)$ are fold points.
These points are nondegenerate if the non-degeneracy assumptions (\ref{NonDeg}) hold:
\begin{align*}
\begin{cases}
\frac{ \partial ^2 f}{ \partial x^2} (x_0^+,y_0^+, \lambda, 0) = -2 x_0^+ = -2 \neq 0 \\
\frac{\partial f}{\partial y}(x_0^+,y_0^+, \lambda, 0) = 1 \neq 0,
\end{cases}
\end{align*}
and equivalently for the other fold point
\begin{align*}
\begin{cases}
\frac{ \partial ^2 f}{ \partial x^2} (x_0^-,y_0^-, \lambda, 0) = -2 x_0^- = 2 \neq 0 \\
\frac{\partial f}{\partial y}(x_0^-,y_0^-, \lambda, 0) = 1 \neq 0.
\end{cases}
\end{align*}
Therefore, the two fold points are non-degenerate.
Futhermore, it can be checked if a fold point is generic. It then has to satisfy the transversality condition $g(x_0,y_0,0) \neq 0$.
The two fold points considered here are generic, since 
\begin{align*}
 g(x_0^+,y_0^+,0)= -1 \neq 0 \\
g(x_0^-,y_0^-,0)= 1 \neq 0.
\end{align*}

Now we know that the Van der Pol System displays Relaxation Oscillations and that normal hyperbolicity of the system breaks down at the fold points. Fenichel Theory can be applied for regions that are not in the neighbourhood of the fold points. However, a different approach has to be employed for the analysis of the dynamics around the folds.\\


In order to analyse a fold point it is convenient to transform the Van der Pol system using a coordinate transformation that satisfies the following: 
\begin{equation} \label{FoldConditions}
    \begin{cases}
        &\folds=(0,0) \textrm{ is a fold point,}\\
        &\pd{^2f}{x^2}(0,0,0)>0\\
        &\pd{f}{y}(0,0,0)<0\\
        & g(0,0,0) <0.
    \end{cases} 
\end{equation}

\subsection{Transformation of the Van der Pol System}\label{sex: mapping} 
In order to analyse the system at the fold points, one fold point at $(x_0^+,y_0^+)=(1,-\dfrac{2}{3})$ is considered, and the further analysis is identical for the second fold point $(x_0^-,y_0^-)$ with a slightly different coordinate transformation. 
The aim is to find a coordinate transformation that satisfies the conditions in (\ref{FoldConditions}). The proposed transformation is $(x,y)\to (1-\Tilde{x},\Tilde{y}-\frac{2}{3})$, which represents a reflection and a translation of the system \st $(x_0^+,y_0^+)$ is mapped to $(0,0)$ - Figure \ref{fig: Transformed System}. \\



Now using the proposed mapping $(x,y)\to (1-\Tilde{x},\Tilde{y}-\frac{2}{3})$ we are able to redefine the fast system (\ref{fastsystem}) in the following way, 
\begin{equation}
    \begin{cases}
        x'=-y+x^2-\dfrac{(x)^3}{3}\\
        y'=\epsilon(x-1),
    \end{cases}
    \label{eq: Fast System}
\end{equation}
where the tilde has been dropped on $x \ \text{and} \ y$ for convenience. The slow system (\ref{slowsystem}) is redefined as
\begin{equation}\label{eq: Slow System}
    \begin{cases}
        \epsilon x'=-y+x^2-\dfrac{(x)^3}{3}\\
        y'=(x-1),
    \end{cases}
\end{equation}
using the normal rescaling of time.
These two systems will be used throughout the following analysis of the generic fold point.
\\
It is readily checked that the coordinate transfomation is correct by evaluating (\ref{FoldConditions}) for the transformed system. It is clear to see that $\folds=(0,0)$, and differentiation of $f$ yields $\pd[2]{f}{x}(0,0,0)=2>0$ and $\pd[1]{f}{y}(0,0,0)=-1<0$. Furthermore, $g(0,0,0) = -1 <0$. Therefore, the new system of equations posesses the required qualities.

\subsection{Reduced Dynamics}

In order to determine the reduced dynamics on the critical manifold $S$, equation (\ref{eq: Slow System}) in the limit $\epsilon\to 0$ is considered which yields the following system,
\begin{align}
\begin{cases}
    &0=f(x,y,0)=-y+x^2-\dfrac{x^3}{3}\\
        &\dot{y}=g(x,y,0)=0 \label{eq: reduced g}
\end{cases}
\end{align}

which is the reduced problem \citep{Kuehn}. 
The critical manifold is then defined as 
\begin{align}
S= \{ (x,y) : f(x,y,0)=0 \} = \left\{ (x,y) : y = x^2-\dfrac{x^3}{3}\right \},
\end{align}
which is an S-shaped curve. 
Since the flow on $S$ is determined by $\dot{y}$, it can be seen that since the sign of $g$ is negative in the neighbourhood of the fold point $(0,0)$, the slow flow on $S$ is directed towards the fold point.

The two  fold points $(x_0^\pm,y_0^\pm)$ coincide with the extrema of the cubic function  $ \phi(x) = y = x^2-\dfrac{x^3}{3}$.
Then using the chain rule, the second equation of (\ref{eq: reduced g}) is  \citep{krupa2001},
\begin{equation}
    \phi_x(x)\dot{x}=g(x,\phi(x),0).
    \label{eq: general reduced}
\end{equation}
Rearranging this gives an expression for the dynamics in $x$ on $S$.
We find that $\phi(x)=x^2-\dfrac{x^3}{3}$, where the derivative \wrt $x$ gives $\phi_x(x)=2x-x^2$.
Therefore (\ref{eq: general reduced}) becomes 
\begin{align*}
	\dot{x} = \frac{g(x,\phi(x),0)}{ \phi_x(x)} = \frac{ x-1}{2x-x^2} =\frac{ x-1}{x(2-x)}.
\end{align*}
This calculation confirms that the fold points at $x=0$ and $x=2$ are singularities of the reduced system. Therefore, no conclusions about the dynamics of $x$ can be made at the fold points. Different methods will have to be developed in order to overcome this.

\subsection{Canonical Form}
In order to simplify the analysis below, it us useful to rewrite the dynamical system in canonical form.
\begin{equation}
    \begin{aligned}
        &x'=-y+x^2-\frac{x^3}{3}=-y+x^2+h(x) \\
        &y'=\epsilon(x-1)
    \end{aligned}
    \label{eq: canonical}
\end{equation}
There is ample reasoning for doing this. The canonical form has been studied in great detail, allowing us to make comparisons and to avoid excess computation, as seen in \citet{krupa2001} paper on Extending Geometric Singular Perturbation Theory.  Note that the first equation has, locally, the shape of the parabola $y= x^2$, which reflects the consideration of the fold point $(0,0)$, which is locally the minimum of a parabola.


\subsection{Extended System}
The canonical system (\ref{eq: canonical}) is then extended to three dimensions by considering $\epsilon'=0$. 
\begin{equation} \label{extended FS}
    \begin{aligned}
        &x'=-y+x^2+h(x) \\
        &y'=\epsilon(x-1)\\
        &\epsilon'=0.
    \end{aligned}
\end{equation}


Analysing the stablility of the three dimensional system, three eigenvalues can be found by considering the Jacobian matrix: 
\begin{equation} 
    J=\begin{vmatrix} 2x-x^2 & -1&0 \\ 0 & 0&0\\0&0&0\end{vmatrix}
    \label{eq: Eigenvalues}
\end{equation}
This is an upper triangular matrix and hence $(\lambda_1.\lambda_2,\lambda_3)=tr(J)= (2x-x^2,0,0)$. Therefore, at the fold points, where $x=0$ or $x=2$, $\lambda_i=0 \ \text{for} \ i=1,2,3$ . Therefore, there exists a zero eigenvalue on $S$ at the fold points. At these points $S$ is not normally hyperbolic.
The critical manifold has to be divided as follows:
\begin{align*}
S^a &= \{ (x,y): y = x^2-\frac{x^3}{3}, x< 0 \} \cup  \{ (x,y): y = x^2-\frac{x^3}{3}, x>2 \} \\
S^r &= \{ (x,y): y = x^2-\frac{x^3}{3}, 0< x< 2 \},
\end{align*}
such that $S^a \cup S^r \cup \{0\} \cup \{2\} = S$.
The manifolds $S^a_0$ and $S^r_0$ are normally hyperbolic everywhere and Fenichel's Theorem (\ref{Fenichel}) can be applied in order to conclude the persistence of the manifold as slow manifolds $S^a_\epsilon$ and $S^r_\epsilon$. At the points $\{0\}$ and $\{2 \}$ the normal hyperbolicity is not given, since the eigenvalue associated to $S$ is zero at these points.
++++++Note: technically the paper mentions here the centre manifold M, stressing the three dimensionality of the problem. if we want that it can be added+++++++++
\\
\\
The problem that the Van der Pol System provides now is the analysis at the fold points.
In the analysis of the reduced system it became apparent that the fold points are singularities of the reduced flow on $S_0$, and therefore the dynamics in the singular limit cannot be determined. Furthermore, Fenichel Theory does not apply at the folds because normal hyperbolicity breaks down at these points, as discussed above. Therefore, even if the dynamics around the folds in the singular limit was known, no conclusions could be drawn for the perturbed system with $S_\epsilon$.
Alternative methods have to be employed to describe the dynamics on the fold points in the singular limit and furthermore to be able to conclude the dynamics of the full system at the fold points from this analysis.
The method considered for analysis is called the Blow-Up Method and is considered in the following section.























 