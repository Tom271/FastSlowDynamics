We have shown that to fully explain the dynamics of a fast-slow system, standard
Geometric Singular Perturbation Theory is not sufficient, due to the lack of
normal hyperbolicity at some points. To remedy this, these points were treated using Blow-Up analysis,
so that `enough hyperbolicity' was gained to apply GSPT. This allowed a full
explanation of the global dynamics in the system, as long as there were no
bifurcations present. In the neighbourhood of $\lambda=0,2$, a Hopf bifurcation
was shown to occur, fundamentally changing the dynamics and introducing small
amplitude oscillations as opposed to the large amplitude relaxation oscillations
seen in other cases. This was caused by a trajectory existing that joined the
attracting manifold to the repelling manifold. The manifold splits under perturbation in either $\lambda$ or phase space. The nature of the split determines the amplitude of the oscillations, and the dynamics around the fold point leading to canards with or without heads. In the planar case, these canard solutions only occur in an
exponentially small region of the fold point and so an additional slow variable was
introduced to produce more persistent canard trajectories. We then showed how in
certain cases this leads to mixed mode oscillatons and even chaotic systems. \\

This approach is widely applicable to fold points in the planar case and folded
nodes in the three dimensional case. However, it does not complete the analysis
of fast-slow systems. If a Hopf bifurcation occurs close to a folded saddle of type II,
the two interact in a way that is not yet fully understood. Further extension is
possible in simulating such fast-slow systems as their stiffness and small scale
pushes the limits of standard numerical methods. \\  

Mixed mode oscillations are an important area of study, as they model a wide range
of biological phenomena. Interesting avenues to be pursued in this area include application of the theory to
the Hodgkin-Huxley model of coupled neurons and the Koper model, see for example \cite{MMO}.
