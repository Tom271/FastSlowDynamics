++++Rename chapter: 'The Blow-Up Method'++++++++

In order to apply the Blow-Up Method to the fold point at the origin, we focus on a neighbourhood $U$ around the fold point $(0,0)$. 
The neigbourhood $U$ is small enough, such that $g(x,y, \epsilon) \neq 0$ in $U$, and we can define sections in $U$, as follows:
\begin{align*}
&\Delta ^{in} = \{ (x, \rho^2), x \in I \} \\
&\Delta ^{out} = \{ (\rho, y), y \in \mathbf{R} \},
\end{align*}
where $I \subset \mathbf{R}$. Now $\Delta^{in}$ is traverse to $S^a$, while $\Delta^{out}$ is traverse to the fast flow. This enables us to monitor the incoming trajectories from the attracting branch of $S$ and the trajectories leaving $U$ in the direction of the fast flow.
Then a function $\pi : \Delta^{in} \to \Delta^{out}$ can be defined, called the transition map, which describes how the trajectories passing through $\Delta^{in}$ are mapped onto the outgoing flow in $\Delta^{out}$.  
The following theorem describes the behaviour of the flow under $\pi$ and a sketch of the proof will be given at the end of this section: ++++last statement not precise+++

\begin{theorem}[\citealp{krupa2001}] \label{transition map theorem}
Under the assumptions made in this section, there exists $ \epsilon_0 >0$ such that the following assertions hold for $\epsilon \in (0, \epsilon_0]$:\\
1. The manifold $S_\epsilon^a$ passes through $\Delta^{out}$ at a point $(\rho, h(\epsilon))$, where $h(\epsilon) = O(\epsilon^{2/3})$.\\
2. The transition map $\pi$ is a contraction with contraction rate $O(e^{-c/\epsilon})$, where $c$ is a positive constant.
\end{theorem}
This means that the trajectories that enter $U$ through $\Delta^{in}$, will be funneled into a smaller section of $\Delta^{out}$ and therefore we are guaranteed to observe the trajectories that enter through $\Delta^{in}$ in $\Delta^{out}$.

Now we are in the position to describe the Method of Blow-Up Transformations in the neighbourhood $U$.

\subsection{Coordinate Transformation}
We first need to transform the extended system (\ref{extended FS}) with respect to the time variable and the space variables. This coordinate transformation is called the Blow-Up Transformation because the degenerate fold point $(0,0)$ (eigenvalue 0, refer to extended system) is regarded as a sphere of radius $r=0$. By rescaling the space variables with respect to different weights of $r$,
\begin{subequations}
    \begin{align}
        &x=\Bar{r}\Bar{x}  \label{sys: blow up trans x}\\
        &y=\Bar{r}^2\Bar{y} \label{sys: blow up trans y}\\ 
        &\epsilon=\bar{r}^3\bar{\epsilon} \label{sys: blow up trans z},
    \end{align}  
    \label{sys: blow up trans}
\end{subequations}
further analysis of the fold is possible. ++(vague....)++
+++++ If time and space permit, an analysis of the space B and coordinate map would be good... p.291 krupa++++

Instead of analysing the sphere in polar coordinates, in order to maximise computational efficiency, the rest of this analysis is done using charts, which are introduced in the next section.

\subsection{Charts} +++++++++++++Include chart picture from book early this section+++++++++++++++++++
In terms of the blown up fold point, a sphere denoted by $B$, charts are projections of regions of $B$ onto a two dimensional plane. 
We introduce three charts $ K_1,K_2,$ and $K_3 $. Chart $K_2$ is the two dimensional projection covering the upper half plane of $B$. However, as points on the equator of $B$ are approached on $K_1$, we tend to infinity to infinity.
These regions however, are of immense interest, since they are points of incoming and outgoing trajectories. (better+++)
Therefore, charts $K_1$ and $K_3$ are introduced, covering the regions of interest on the equator. This is represented in Figure ++++++ (blow up figure)++

The charts are defined by setting each of the variables of the extended system to $1$ in turn, giving $ \bar{y}=1, \ \bar{\epsilon}=1, \ \bar{x}=1 $. Substituting these into Equations (\ref{sys: blow up trans x}), (\ref{sys: blow up trans y}) and (\ref{sys: blow up trans z}) respectively gives, 
\begin{subequations} \label{ sys: K1K2K3}
	\begin{align}
	&x=r_1x_1, \ y=r_1^2, \ \epsilon=r_1^3\epsilon_1,\\
	&x=r_2x_2, \ y=r_2^2y_2, \ \epsilon=r_2^3 \label{sys: K2}\\
	&x=r_3, \ y=r_3^2y_3, \ \epsilon=r_3^3\epsilon_3\label{sys:K3}
	\end{align}
\end{subequations}
where $ (x_i,r_i,\epsilon_i)\in\mathbf{R}^3 $ for $ i=1,2,3 $, and the equations correspond to the charts in numerical order \citep{krupa2001}. 
With this setup, we can consider the individual charts in turn, analyse the dynamics on the individual charts, and then join the gathered information into a global view on the dynamics in $U$.
We start with $K_2$,  because it holds the most information and the flow is the analysed more readily than in the other two charts. 
The remaining question is how the transition between the three charts and the connection to the global dynamics is made after finishing the individual analysis.
This is done via a coordinate change, derived by using equations (\ref{ sys: K1K2K3}) and (\ref{sys: blow up trans}), and the results are summed up in the following Lemma:
\begin{lemma}
Let $\kappa_{12}$ denote the change of coordinates from $K_1$ to $K_2$. Then $\kappa_{12}$ is given by \\
\begin{equation}
x_2 = x_1 \epsilon_1^{-1/3},  y_2 = \epsilon_1^{-2/3}, r_2= r_1\epsilon_1^{1/3},
\end{equation}
for $\epsilon_1 >0$,
and $\kappa_{12}^{-1}$ is given by
\begin{equation}
x_1 = x_2y_2^{-1/2}, r_1 = r_2 y_2^{1/2}, \epsilon_1= y_2^{-3/2},
\end{equation}
for $y_2>0$.
Let $\kappa_{23}$ denote the change of coordinates from $K_2$ to $K_3$. Then $\kappa_{23}$ is given by
\begin{equation}
r_3 = r_2x_2, y_3= y_2x_2^{-2}, \epsilon_3 = x_2^{-3},
\end{equation}
for $x_2>0,$
and $\kappa_{23}^{-1}$  is given by
\begin{equation}
x_2 = \epsilon_3^{-1/3}, y_2 = y_3\epsilon_3^{-2/3}, r_2= r_3 \epsilon_3^{1/3},
\end{equation}
for $\epsilon_3>0$.
\end{lemma}

Furthermore, transition maps $\Pi_i, i \in 1,2,3$  are defined in each section, describing how the trajectories coming in and out of each chart. These are combined in the final part of this section to give the proof of Theorem \ref{transition map theorem}, and to relate the results of the blow up method back to the original transition map $\pi$.

\subsection{Dynamics in \texorpdfstring{$K_2$}{K2}} \label{sec: VDP K_2}
To be able to consider chart $ K_2$, the transformation presented in Equation (\ref{sys: K2}) is applied to the extended system (\ref{extended FS}). Furthrmore,a time rescaling ($ t_2=r_2t $) is applied to desingularise the system. This results in:
\begin{align}
&\od{}{t}(r_2x_2)=r_2^2\od{x_2}{t}=-y_2+x_2^2-\dfrac{x_2^3r_2}{3},\\
&r^3_2y'_2=r^3_2(-1+r_2x),\\
&r'_2=0,
\end{align}
noting that $ \od{r}{t_2}=\od{t}{t_2}\od{r}{t}=\frac{1}{r_2}\od{r_2}{t} $ .  Now dividing through by $ r^2_2 $ and $ r^3_2 $ respectively for each equation and grouping $O(r_2)$ terms we get,
\begin{equation}
	\begin{aligned}
		&x'_2=x^2_2-y_2+O(r_2),\\
		&y'_2=-1+O(r_2),\\
		&r'_2=0.
	\end{aligned}
\end{equation}
Then, considering $r_2=0$ and neglecting the $O(r_2)$ terms results in:
\begin{equation}
	\begin{aligned}
		&x'_2=x^2_2-y_2,\\
		&y'_2=-1,\\
	\end{aligned}
\end{equation}
which are the well known Riccati equations- see \citet{Riccati}.
Some known results about the Riccati equations can be summarised as follows:


\subsection{Dynamics in \texorpdfstring{$K_1$}{K1}}

\subsection{Dynamics in \texorpdfstring{$K_3$}{K3}}
Similarly to $K_1$ and $K_2$, the system can be transformed using Equation \ref{sys:K3}.
\begin{align*}
\od{r_3}{t_3}&=r_3F(r_3,y_3,\epsilon_3)\\
\od{y_3}{t_3}&=\epsilon_3(r_3-1)-2y_3F(r_3,y_3,\epsilon_3)\\
\od{\epsilon_3}{t_3}&=-3\epsilon_3F(r_3,y_3,\epsilon_3)
\end{align*}
where $F(r_3,y_3,\epsilon_3)=\left(1-y_3-\frac{r_3}{3}\right)$
\newpage