

\begin{theorem}[\cite{krupa2001}]
	Assume that system (3.1) satisfies the defining non-degeneracy conditions (Equations \ref{eq: canard sing condition} and \ref{eq: canard non-d condition}) of a canard point. Assume that the solution $ x_0(t) $ of the reduced problem connects $ S_a $ to $ S_r $. Then there exists $ \epsilon_0 > 0 $ and a smooth function $\lambda_c(\sqrt{\epsilon})$ defined on $ [0, \epsilon_0] $ such that for $\epsilon \in (0, \epsilon_0)$ the following assertions hold:
	\begin{itemize}
		\item $ \pi(q_{a,\epsilon})=q_{r,\epsilon} $ iff $ \lambda=\lambda_c(\sqrt{\epsilon}) $.\\
		\item The function $ \lambda_c $ has the expansion
		 \begin{equation*}
			\lambda_c(\sqrt{\epsilon})=-\epsilon(\frac{a_1+a_5}{2}+\frac{A}{8})+O(\epsilon^\frac{3}{2}).
			\end{equation*}
			\item The transition map $\pi$ is defined only for $\lambda$ in an interval around $ \lambda_c(\sqrt{\epsilon}) $ of width $ O(\exp(-\frac{c}{\epsilon})) $ fo some $ c>0 $.
			$$ \pd{}{\lambda}(\pi(q_{a,\epsilon})-q_{r,\epsilon})|_{\lambda=\lambda_c(\sqrt{\epsilon})}>0  $$
		
	\end{itemize}
\label{Theorem 3.1}
\end{theorem}
\begin{theorem}[Canards in $ \Re^3 $ \citep{MMO}]\label{thm: canards in R3}
	For slow-fast systems (Equation \ref{eq: fs singularity system}) with $ \epsilon>0 $ sufficiently small the following holds:\begin{itemize} 
		\item  There are no maximal canards generated by a folded focus. For a folded saddle the two singular canards $ \bar{\gamma_{1,2}} $ perturb to maximaal canards $ \gamma_{1,2} $.
		\item  For a folded node let $\mu=\frac{\sigma_w}{\sigma_s} <1$. the singular canard $ \bar{\gamma_{s}} $ (``the strong canard'') always perturbs to a maximal canard $ \gamma_{s} $. If $ \mu^{-1}\not \in \mathbb{N} $, then the singualr canard $ \bar{\gamma_{w}} $ (``weak canard'') also perturbs to a maximal canard. We call $ \gamma_{s} $ and $ \gamma_{w} $ primary canards.
		\item For a folded node suppose $ k>0 $ is an integer such that $ 2k+1<\mu^{-1} <2k+3$ and $ \mu^-1\neq 2(k+1) $. Then, in addition to $ \gamma_{s,w} $ there are k other maximal canards, which we call secondary canard.
		\item The primary weak canard of a node undergoes a trancritical bifurcation for odd $ \mu^{-1}\in\mathbb{N} $ and a pitchfork bifurcation for even $ \mu^{-1}\in\mathbb{N} $
	\end{itemize}
\end{theorem}