

\begin{theorem}[\cite{krupa2001}]
	Assume that system (3.1) satisfies the defining non-degeneracy conditions (Equations \ref{eq: canard sing condition} and \ref{eq: canard non-d condition}) of a canard point. Assume that the solution $ x_0(t) $ of the reduced problem connects $ S_a $ to $ S_r $. Then there exists $ \epsilon_0 > 0 $ and a smooth function $\lambda_c(\sqrt{\epsilon})$ defined on $ [0, \epsilon_0] $ such that for $\epsilon \in (0, \epsilon_0)$ the following assertions hold:
	\begin{itemize}
		\item $ \pi(q_{a,\epsilon})=q_{r,\epsilon} $ iff $ \lambda=\lambda_c(\sqrt{\epsilon}) $.\\
		\item The function $ \lambda_c $ has the expansion
		\begin{itemize}[label={}]
			\begin{equation*}
			\lambda_c(\sqrt{\epsilon})=-\epsilon(\frac{a_1+a_5}{2}+\frac{A}{8})+O(\epsilon^\frac{3}{2}).
			\end{equation*}
		\end{itemize}
			\item The transition map $\pi$ is defined only for $\lambda$ in an interval around $ \lambda_c(\sqrt{\epsilon}) $ of width $ O(\exp(-\frac{c}{\epsilon})) $ fo some $ c>0 $.
			\item $$ \pd{}{\lambda}(\pi(q_{a,\epsilon})-q_{r,\epsilon})|_{\lambda=\lambda_c(\sqrt{\epsilon})}>0  $$
		
	\end{itemize}
\label{Theorem 3.1}
\end{theorem}