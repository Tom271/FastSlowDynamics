
Many figures in this document? were produced using MATLAB, for example: fig +++++++. In this appendix, we will give a brief tutorial on their production.

\begin{lstlisting}[style=Matlab-editor]
function vdpPlot= vdpPlot(epsilon,lambda,tSpan,initialValue,xyPlot,tPlot)
% VDPPLOT Plot solutions to the van der Pol equation
%Plot solutions to the vdP equation with mu=m over the interval [0 
% tSpan] from the point initialValue in the x-y plane.
assert(epsilon>=0, 'm must be greater than zero')
tolerance=1/10;
if(epsilon<tolerance)
[t,x]=ode15s(@(t,x) vdp(t,x,epsilon,lambda),[0 tSpan],initialValue);
end

if(epsilon>=tolerance)
[t,x]=ode45(@(t,x) vdp(t,x,epsilon,lambda),[0 tSpan],initialValue);
end

fold1=[0 0];
fold2=[2,4/3];
equil=[1,2/3];
if(xyPlot==1)
nullx=linspace(-1,3);
nully=-nullx.^3/3.+nullx.^2;
null2y=nullx;
fig=figure(1);
set(gcf,'color','white')    
%'units','normalized','outerposition',[0 0 1 1],

[quivx,quivy]=meshgrid(-1:0.1:3);
dquivx=-quivy-quivx.^3/3+quivx.^2;
dquivy=epsilon*(-lambda+dquivx);

for j = 1:length(x(:,1))
hold on
title(sprintf('Epsilon= %.3f, Initial point =(%d,%d)',epsilon,initialValue))
axis( [-1,3, -1,3] )
quiver(quivx,quivy,dquivx,dquivy,'Color','b')
plot(nullx,nully,'color','black','LineWidth',1.5)
plot(fold1(1),fold1(2),'g-s','MarkerFaceColor','g','MarkerSize',10)
plot(fold2(1),fold2(2),'g-s','MarkerFaceColor','g','MarkerSize',10)
plot(equil(1),equil(2),'b-s','MarkerFaceColor','b','MarkerSize',10)
plot(ones(1,length(null2y)),null2y,'--','LineWidth',1.5,'Color',1/255*[150,150,150])
plot(x(1:j,1),x(1:j,2),'-o','Color','r','MarkerIndices',1:(j-1):j);
hold off
Mv(j) = getframe(fig);
children = get(fig, 'children');

if(j~=length(x(:,1)))
delete(children(1));
end 

end
movie(Mv,1);
v = VideoWriter('vdP.mp4','MPEG-4');
v.Quality=100;
v.FrameRate=30;

open(v)
writeVideo(v,Mv)
close(v)

end
if(tPlot==1)
figure();
set(gcf,'color','white')
plot(t,x(:,1),t,x(:,2))
title('Time Plot of the vdP Oscillator')
end

end
function vdp =vdp(t,x,epsilon,lambda)
vdp=[-x(2)-x(1).^3/3+x(1).^2;epsilon*(-lambda+x(1))];
end 




\end{lstlisting}