
Many figures in this document? were produced using MATLAB, for example: fig +++++++. In this appendix, we will give a brief tutorial on their production. Fast-slow systems like the ones studied here are a classic example of \emph{stiff} ODEs\footnote{Indeed, the MATLAB documentation for its stiff solver, \texttt{ode15s}, uses the Van Der Pol equation as it's example.}. 
\begin{definition}[Stiffness Ratio]
	Consider $\dot{x} = F(x)$ where $x\in \mathbf{R}^n, F\in C^r(\mathbf{R}^n,\mathbf{R}^n)$. Let 
	$$ x' = Ax,\quad A\in \mathbf{R}^{n \times n}  $$
	denote its linearisation. Suppose all the eigenvalues $\lambda_j$  of $A$ have negative real parts. Then the \emph{stiffness ratio}, $\mu$ is defined as
	$$ \mu :=\frac{\max_j(\mathrm{Re}(\lambda_j))}{\min_j(\mathrm{Re}(\lambda_j))} $$
	If $\mu$ is large, the system is called \emph{stiff}.
\end{definition}
Stiffness is not a well-defined concept, it can be seen as a general term for a set of equations which are difficult to solve numerically to a high level of accuracy. Throughout this section we will consider the general problem above as an initial value problem.

$$ \begin{cases}
\dot{x} = F(x)\\
x(T_0)=x_0
\end{cases}$$
As before, $x\in \mathbf{R}^n$ and $ F\in C^r(\mathbf{R}^n,\mathbf{R}^n)$. To solve such a system numerically, time must be discretised. Using standard notation, let $h$ be the time step between points on the solution. To differentiate between the continuous solution $x(t)$ and the discretised solution, we denote the latter by $x(t_j)=x_j$. Here $t_j = T_0+jh$. As a first example, consider the modified Euler method.

$$ x(t_{n+1})=x(t_n)+hF\left( x(t_n) +\frac{1}{2}F(x(t_n))\right)$$
Or, in the more compact notation, 
$$x_{n+1}=x_n+hF\left( x_n +\frac{1}{2}F(x_n)\right)$$
This is a simple method and provides a starting point in considering error between true and numerical solutions.\\

The go-to ODE solver in MATLAB is \texttt{ode45}. This function uses the Dormand-Prince Runge-Kutta method, an explicit single-step formula. The Runge-Kutta method (RK4) is similar to the explicit Euler method in that it calculates the next point ($x_{n+1}$) using only its current value ($x_n$). Unlike the Euler method however, it yields much lower error by using a better approximation of the derivative at points in between $x_n$ and $x_{n+1}$ as opposed to only the derivative at the initial point. The Runge-Kutta method uses the following relation.


$$ x_{n+1} = x_n +\frac{1}{6} h \left(k_1+2k_2+2k_3+k_4\right)$$
where
\begin{align*}
k_1 &= F(x_n)\\
k_2 &= F\left(x_n+\frac{1}{2}hk_1\right)\\
k_3 &= F\left(x_n +\frac{1}{2}hk_2\right)\\
k_4 &= F\left(x_n + hk_3\right)
\end{align*}
The Runge-Kutta family of solvers are ubiquitous in numerical analysis, and most methods can be categorised as belonging to this set of methods. Even the simplest, the explicit Euler scheme, is a RK method.

\begin{ex}
		Fast slow system example 
		$$\begin{cases}
		\dot{x}=-\frac{x}{\epsilon}\\
		\dot{y}=-y
		\end{cases}$$
		
		Test on RK4, mod-Euler and \texttt{ode15s}? Intro BDF? Check sec8 MMO.
\end{ex}
