
\citet{krupa2001} discusses why we should make a transformation for our system. We find that if we map $\folds=(0,0)$ we are then able to simplify our non-degeneracy conditions - as shown in Equation \ref{non-degeneracy} \citep{krupa2001}.
\begin{equation}
    \begin{cases}
        &\folds=(0,0)\\
        &\pd{^2f}{x^2}(0,0,0)>0\\
        &\pd{f}{y}(0,0,0)<0
    \end{cases} 
    \label{non-degeneracy}
\end{equation}

\subsection{Mapping Transformation}\label{sex: mapping}
To be able to continue with our analysis we should consider the transformations. We will first only consider the case for our fold point at $(x_0^+,y_0^+)=(1,-\dfrac{2}{3})$. We wish to map $(x,y)\to (1-\Tilde{x},\Tilde{y}-\frac{2}{3})$ which reflects and translates our system \st our fold points are now mapped to $(0,0)$ - Figure \ref{fig: Transformed System}. 

\begin{figure}[h]
    \centering
    %\includegraphics{}
    \caption{Our transformed system.}
    \label{fig: Transformed System}
\end{figure}
Now that we have made our transformation we can check our non-degeneracy conditions. However, before we continue we should check that our the sign of the derivative is conserved through the transformation. We can do this by using the chain rule as follows, 

\begin{equation}
    \od{x}{t}=\od{\Tilde{x}}{x}\od{x}{t}=-\od{\Tilde{x}}{t}. \label{chain rule}
\end{equation}
Now using Equation \ref{chain rule} and our new mapping ( $(x,y)\to (1-\Tilde{x},\Tilde{y}-\frac{2}{3})$) we are able to define our new Fast System in the following way, 
\begin{equation}
    \begin{cases}
        x'=-y+x^2-\dfrac{(x)^3}{3}\\
        y'=\epsilon(x-1)
    \end{cases}
    \label{eq: Fast System}
\end{equation}
where we note that we have dropped the tilde on $x \ \text{and} \ y$ for convenience. \\

\subsection{Non-degeneracy Conditions}
\begin{equation}
    \begin{cases}
        &\folds=(0,0)\\
        &\pd[2]{f}{x}(0,0,0)>0\\
        &\pd{f}{y}(0,0,0)<0
    \end{cases} 
    \tag{\ref{non-degeneracy}}
\end{equation}
Now that we have constructed our transformed system, we are now able to check our non-degeneracy conditions - Equation \ref{non-degeneracy}. It is clear to see that $\folds=(0,0)$ by the mapping we defined in Section \ref{sex: mapping}. Following this we are able to check our other non-degeneracy conditions conform to our new mapping. The differentiation is easily seen from Equation \ref{eq: Fast System} which yields that $\pd[2]{f}{x}(0,0,0)=2>0$ and $\pd[1]{f}{y}(0,0,0)=-1<0$, confirming our assumptions.


\subsection{Reduced Dynamics}
The next progression for our system is to consider the reduced dynamics within our system. To do this we consider Equation \ref{slowsystem} and take the $\epsilon\to0$ which yields the following system,
\begin{subequations}
    \begin{align}
    &0=f(x,y,0)=-y+x^2-\dfrac{x^3}{3}\\
        &\dot{y}=g(x,y,0)=0 \label{eq: reduced g}
    \end{align}
\end{subequations}
which is known as the slow subsystem \citep{Bible}. We are then able to compute the reduced flow by computing  \citep{krupa2001},
\begin{equation}
    \phi_x(x)\dot{x}=g(x,\phi(x),0),\text{ for }y=\phi(x).
    \label{eq: general reduced}
\end{equation}
We find that $\phi(x)=x^2-\dfrac{x^3}{3}$ where the derivative \wrt $x$ gives $\phi_x(x)=2x-x^2$. Now it is clear that we will have a singularity at $x=0$, by Equation \ref{eq: general reduced}, thus we will find that our system blows up ($\dot{x}\to\infty$) which motivates the process that will follow. \\ \\
%note that the sign g(0,0,0) determines the flow direction 
%g(0,0,0)<0 as expected in our example.
After finding our reduced system we are able to determine the way in which it flows by considering the sign of $g(0,0,0)$. We see from Equation \ref{eq: reduced g} that $g<0$ then we have that our flow is directed towards the fold points $(0,0)$. To continue with our analysis we first need to define Fenichel's theorems.

\subsection{Fenichel and Standard Theory}
\textbf{THEOREM}

\subsection{Extended System}
Now that we have established the above theorems then we need to consider the extended system \st $\epsilon'=0$, thus $\epsilon=\text{\textit{const}}$. By considering this system we will be able to consider a blow up (magnification) around our fold point in three dimensions.  
\begin{figure}[h]
    \centering
%    \includegraphics{}
    \caption{Blown up system}
    \label{fig: blow up}
\end{figure}
From Figure \ref{fig: blow up} we can consider the stability of our fold point. To do this we are able to establish the following determinant,
\begin{equation} 
    A=\begin{vmatrix} 2x-x^2-\lambda & -1&0 \\ 0 & -\lambda&0\\0&0&-\lambda \end{vmatrix}=\lambda^2(2x-x^2-\lambda).
    \label{eq: Eigenvalues}
\end{equation}
However, a far easier approach is to note that our matrix is an upper triangular matrix which we can take the eigenvalues directly from Equation \ref{eq: Eigenvalues} \st $(\lambda_1.\lambda_2,\lambda_3)=tr(A)$. Then we can clearly see that for our fold points $\lambda_i=0 \ \text{for} \ i=1,2,3$ and for any $x\neq0$ we have $\lambda_1=x(2-x)$ and $\lambda_2=\lambda_3\equiv0$. As a result we can see that we are forced to blow up our system around our fold point as our steady states are non-hyperbolic whereas outside of this we find that we have one hyperbolic steady state.
%Now we can see that we have three non-hyperbolic eigen values when we consider our system at (0,0,0), 


\subsubsection{Canonical Form}
Now that we have established our reduced system we are able to rewrite it in canonical form - Equation \ref{eq: canonical}.
\begin{equation}
    \begin{aligned}
        &x'=-y+x^2-\frac{x^3}{3}=-y+x^2+h(x) \\
        &y'=\epsilon(x-1)
    \end{aligned}
    \label{eq: canonical}
\end{equation}
There is ample reasoning for doing this. This is because we find that the canonical form has been studied in great detail allowing us to make comparisons and to avoid excess computation. This then allows us to follow an analogous approach to the many papers associated with this topic - as seen in \citet{krupa2001} paper on Extending Geometric Singular Perturbation Theory.
