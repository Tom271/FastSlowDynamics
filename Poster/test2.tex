\documentclass[15pt, a0paper, portrait]{tikzposter}
\usepackage[utf8]{inputenc}
 
\title{Fast-Slow Dynamics}
\author{Jonna, Kieran, Tom }
\date{14/12/2018}
%\institute{MIGSAA}
 
\usepackage{blindtext}
\usepackage{comment}
\usepackage{amsmath}
%\usepackage{proj1}
\usetheme{Board}
 
\begin{document}
 
\maketitle
 
\block{~}
{

Fast- Slow systems are systems of differential equations that can be viewed on two different time scales, which are separated by a parameter.
These systems are generally of the form
\begin{equation*} 
\begin{cases}
x' &=\frac{dx}{dt}= f(x,y,\lambda, \epsilon),\\
y' &= \frac{dy}{dt}= \epsilon g( x,y, \lambda, \epsilon),
\end{cases}\label{FastS}
\end{equation*}
which is known as the fast system.
Using a scaling for the time, $t = \frac{\tau}{\epsilon} $, we find that this can be rewritten as
\begin{align*}
\begin{cases}
\epsilon \dot{x} &= \epsilon \frac{dx}{d \tau} = f(x,y,\lambda, \epsilon),\\
\dot{y} & = \frac{dy}{d \tau} =  g( x,y, \lambda, \epsilon),
\end{cases}\label{SlowS}
\end{align*}
which is called the slow system.
}
     \block{Van der Pol System}
\begin{tabular}{c|c}
%    \column{0.5}

%{
	Fast System:
\begin{equation*}\label{fastsystem}
    \begin{cases} x'=y-\frac{x^3}{3}+x\\
    y'=-\epsilon x,
    \end{cases}
\end{equation*}
Slow System: 
\begin{equation*}\label{slowsystem}
    \begin{cases} \epsilon \dot{x}=y-\frac{x^3}{3}+x\\
    \dot{y}=-x,
    \end{cases}
\end{equation*}
%}
&
%   
%\colorlet{blockbodybgcolor}{green!60}
%\block{Phase Portrait: Van der Pol System}
%{
%\column{0.5}
	 
	\begin{tikzfigure}[h!]
		\centering
		\includegraphics[height=10cm,width=13cm]{Posterpic1.jpg}
		%\caption{The reduced flow where a) $\lambda=0$ and b) $\lambda>0$.}
		%label{fig: Canard Point}
	\end{tikzfigure}
%}
\end{tabular}


\end{document}